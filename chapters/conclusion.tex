\chapter{\ifenglish Conclusions and Discussions\else บทสรุปและข้อเสนอแนะ\fi}

\section{\ifenglish Conclusions\else สรุปผล\fi}

โครงงานนี้ประสบความสำเร็จในการพัฒนาระบบครบทั้ง 3 ส่วนตามวัตถุประสงค์ที่กำหนดไว้ ประกอบด้วย (1) ส่วน web front-end สำหรับผู้ใช้งานทั่วไปและผู้ดูแลระบบ ที่ช่วยให้การเข้าถึงและจัดการข้อมูลเป็นไปอย่างมีประสิทธิภาพ (2) ส่วน mobile application สำหรับการตรวจจับและประมวลผลป้ายจราจรแบบเรียลไทม์ และ (3) ส่วน back-end ที่รองรับการประมวลผลข้อมูล การจัดเก็บ และการเชื่อมต่อระหว่างส่วนต่างๆ ของระบบ แม้ว่าระบบจะสามารถทำงานได้ตามเป้าหมายหลัก แต่ยังพบข้อจำกัดบางประการที่สำคัญ ดังนี้

\begin{itemize}
    \item ความยาววีดีโอต่อ 1 เซสชั่นไม่ควรเกิน 30 นาที
    \item ระบบตรวจจับป้ายอาจมีการทำงานผิดพลาด เนื่องจากความแม่นยำของตัวโมเดลไม่ได้สูงมาก
    \item ความแม่นยำของตำแหน่งป้ายจะขึ้นอยู่กับคุณภาพของโทรศัพท์มือถือที่ใช้งาน
    \item การแสดงผลของป้านในตำแหน่งเดียวกันนั้น จะสามารถแสดงได้แค่ 1 ป้ายในพิกัดนั้นๆ เท่านั้น
    \item หลังจากที่ระบบประมวลผลข้อมูลเสร็จสิ้น ไม่ว่าจะสำเร็จหรือไม่สำเร็จก็ตาม ระบบไม่อนุญาตให้ผู้ใช้งานทำการประมวลผลอีกครั้งได้ จะต้องติดต่อกับผู้ดูแลระบบเท่านั้น
    \item แอพพลิเคชั่นมือถือที่ใช้ในการตรวจจับป้ายจราจรนั้น จะต้องใช้งานบนระบบปฏิบัติการ Android \\เท่านั้น
\end{itemize}

\section{\ifenglish Challenges\else ปัญหาที่พบและแนวทางการแก้ไข\fi}

\begin{itemize}
    \item ความยาววิดีโอที่ไม่ควรเกิน 30 นาที นั้นสืบเนื่องมาจากระบบ auto scale ที่ได้ออกแบบมานั้นมีระบนยะเวลาการทำงานต่อ 1 เซสชั่นประมาณ 1 ชั่วโมง ซึ่งควารมยาวของการประมวลผลจะใช้เวลาเป็น 2 เท่าของความยาววิดีโอ ดังนั้นความยาววิดีโอจึงไม่ควรที่จะเกิน 30 นาที วิธีแก้ไขสามารถทำได้โดยการปรับเปลี่ยนโครงสร้างระบบประมวลผลบน cloud ให้สามารถทำงานได้ดีขึ้น
    \item การทำงานของระบบตรวจจับป้ายยังมีความแม่นยำที่ต่ำ เนื่องจากการเตรียมข้อมูลสำหรับการเรียนรู้ของโมเดลไม่เพียงพอ ดังนั้นวิธีการแก้ไขคือการเพิ่มข้อมูลสำหรับการเรียนรู้ของโมเดลและปรับปรุงโมเดลให้มีความแม่นยำมากขึ้น
    \item การแสดงผลของป้านในตำแหน่งเดียวกันนั้น จะสามารถแสดงได้แค่ 1 ป้ายในพิกัดนั้นๆ เท่านั้น ดังนั้นการแสดงผลของป้ายจึงไม่สามารถที่จะแสดงผลได้ทั้งหมด วิธีการแก้ไขคือการปรับปรุงโค้ตของการแสดงผลของป้ายให้สามารถแสดงผลได้ทั้งหมด
    \item เนื่องจากระบบที่ได้ออกแบบมานั้นมีความซับซ้อนค่อนข้างมาก ดังนั้นการ Deploy ก็จะต้องใช้ความเข้าใจในระบบอย่างลึกซึ้ง ทำให้ระบบนี้ผู้ที่ไม่มีความรู้เฉพาะทางอาจจะไม่สามารถที่จะ Deploy ระบบได้
    \item หน้าเว็บไซต์ที่ใช้ในการแสดงผลป้ายนั้นจะมีการ render จุดของป้ายเป็นจำนวนมากดังนั้น หากคอมพิวเตอร์ที่ใช้งานไม่มีความสามารถในการ render จุดของป้ายเป็นจำนวนมาก จะทำให้หน้าเว็บไซต์ทำงานได้ช้าและไม่ลื่นไหล วิธีการแก้ไขอาจทำได้โดยการปรับปรุงโค้ตของการ render ให้ optimize มากขึ้นเพื่อให้ทำงานได้ในทุก ๆ เครื่อง
    \item ตำแหน่งของป้ายเดียวกันนั้นสามารถที่จะซ้ำกันได้ ในกรรีที่เราขับผ่านถนนเส้นเดิมมากกว่า 1 ครั้ง ดังนั้นการแสดงผลของป้ายจึงไม่สามารถที่จะแสดงผลได้ถูกต้อง 
\end{itemize}

\section{\ifenglish%
Suggestions and further improvements
\else%
ข้อเสนอแนะและแนวทางการพัฒนาต่อ
\fi
}

\begin{itemize}
    \item อนุญาตให้ผู้ใช้สามารถอัพโหลด shapefile ของขอบเขตการทำงานขององค์กรตัวเอง เพื่อความรวดเร็วในการกำหนดของเขตการทำงานขององค์กร
    \item พัฒนาระบบให้สามารถทำงานได้ในระบบปฏิบัติการ iOS ด้วย
    \item พัฒนาแอพพลิเคชั่นให้ตอบรับต่อมาตรฐานของในแต่ละ store เพื่อให้ผู้ใช้งานสามารถที่จะดาวน์โหลดแอพพลิเคชั่นได้ง่ายดายมากยิ่งขึ้น
    \item ปรังปรุงระบบประมวลผลบน cloud ให้สามารถทำงานได้ยาวนานมากยิ่งขึ้น
    \item ปรับปรุงโมเดลให้มีความแม่นยำมากขึ้น
    \item ปรับปรุงโค้ดในส่วนของการแสดงผลของป้ายให้สามารถแสดงผลได้มากกว่า 1 ป้ายในตำแหน่งเดียวกัน
\end{itemize}