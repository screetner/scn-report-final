\chapter{\ifenglish Conclusions and Discussions\else บทสรุปและข้อเสนอแนะ\fi}

\section{\ifenglish Conclusions\else สรุปผล\fi}

โครงงานนี้ประสบความสำเร็จในการพัฒนาระบบครบทั้ง 3 ส่วนตามวัตถุประสงค์ที่กำหนดไว้ ประกอบด้วย (1) ส่วน web front-end สำหรับผู้ใช้งานทั่วไปและผู้ดูแลระบบ ที่ช่วยให้การเข้าถึงและจัดการข้อมูลเป็นไปอย่างมีประสิทธิภาพ (2) ส่วน mobile application สำหรับการตรวจจับและประมวลผลป้ายจราจรแบบเรียลไทม์ และ (3) ส่วน back-end ที่รองรับการประมวลผลข้อมูล การจัดเก็บ และการเชื่อมต่อระหว่างส่วนต่างๆ ของระบบ แม้ว่าระบบจะสามารถทำงานได้ตามเป้าหมายหลัก แต่ยังพบข้อจำกัดบางประการที่สำคัญ ดังนี้

\begin{itemize}
    \item ความยาววีดีโอต่อ 1 เซสชั่นไม่ควรเกิน 30 นาที
    \item ระบบตรวจจับป้ายอาจมีการทำงานผิดพลาด เนื่องจากความแม่นยำของตัวโมเดลไม่ได้สูงมาก
\end{itemize}

\section{\ifenglish Challenges\else ปัญหาที่พบและแนวทางการแก้ไข\fi}

\begin{itemize}
    \item การทำงานของระบบตรวจจับป้ายยังมีความแม่นยำที่ต่ำ เนื่องจากการเตรียมข้อมูลสำหรับการเรียนรู้ของโมเดลไม่เพียงพอ ดังนั้นวิธีการแก้ไขคือการเพิ่มข้อมูลสำหรับการเรียนรู้ของโมเดลและปรับปรุงโมเดลให้มีความแม่นยำมากขึ้น
    \item ความยาววิดีโอที่ไม่ควรเกิน 30 นาที นั้นสืบเนื่องมาจากระบบ auto scale ที่ได้ออกแบบมานั้นมีระบนยะเวลาการทำงานต่อ 1 เซสชั่นประมาณ 1 ชั่วโมง ซึ่งควารมยาวของการประมวลผลจะใช้เวลาเป็น 2 เท่าของความยาววิดีโอ ดังนั้นความยาววิดีโอจึงไม่ควรที่จะเกิน 30 นาที และวิธีการแก้ไขคือการปรับปรุงระบบ auto scale ให้สามารถทำงานได้เร็วขึ้น
\end{itemize}

\section{\ifenglish%
Suggestions and further improvements
\else%
ข้อเสนอแนะและแนวทางการพัฒนาต่อ
\fi
}

ข้อเสนอแนะเพื่อพัฒนาโครงงานนี้ต่อไป มีดังนี้