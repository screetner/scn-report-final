\newcommand{\hbarWidth}{6.6cm}

\newcommand{\horizontalScreenshotWidth}{100mm}
\newcommand{\verticalScreenshotWidth}{40mm}

\newcommand{\pageLoginEmpty}{
    \insertPDFfigure[page=1,width=\verticalScreenshotWidth]
    {resources/mobile-app/page-login-default.png}
    {\ifenglish Login Page \else หน้าเข้าสู่ระบบ \fi}
    {page-login-default}
}

\newcommand{\pageLoginDefault}{\insertPDFfigureManual[page=1,width=\verticalScreenshotWidth][H]{resources/mobile-app/page-login-default.png}{\ifenglish Login Page  \else หน้าเข้าสู่ระบบ \fi}{page-login-default}} 

\newcommand{\pageLoginException}{\insertPDFfigureManual[page=1,width=\verticalScreenshotWidth][H]{resources/mobile-app/page-login-exception.png}{\ifenglish Login Page with Incorrect Credentials \else หน้าเข้าสู่ระบบเมื่อกรอกข้อมูลผู้ใช้ผิด \fi}{page-login-exception}} 

\newcommand{\pageHomeEmpty}{\insertPDFfigureManual[page=1,width=\verticalScreenshotWidth][H]{resources/mobile-app/page-home-empty.png}{\ifenglish Home Page without video sessions \else หน้าแรกโดยไม่มีเซสชันวิดีโอ \fi}{page-home-empty}}

\newcommand{\pageHomePreUpload}{\insertPDFfigureManual[page=1,width=\verticalScreenshotWidth][H]{resources/mobile-app/page-home-pre-upload.png}{\ifenglish Home Page with video sessions \else หน้าแรกโดยมีเซสชันวิดีโอ \fi}{page-home-pre-upload}}

\newcommand{\pageHomeExtendedCard}{\insertPDFfigureManual[page=1,width=\verticalScreenshotWidth][H]{resources/mobile-app/page-home-extended-card.png}{\ifenglish Home Page with extended card \else หน้าแรกโดยมีการ์ดที่ขยาย \fi}{page-home-extended-card}}

\newcommand{\pageHomeUploading}{\insertPDFfigureManual[page=1,width=\verticalScreenshotWidth][H]{resources/mobile-app/page-home-uploading.png}{\ifenglish Home Page with uploading session \else หน้าแรกโดยมีการ์ดที่กำลังอัปโหลด \fi}{page-home-uploading}}

\newcommand{\pageHomeUploadingNotificaiton}{\insertPDFfigureManual[page=1,width=60mm][H]{resources/mobile-app/page-home-uploading-notification.png}{\ifenglish Uploading notification \else การแจ้งเตือนการอัปโหลด \fi}{page-home-uploading-notification}}

\newcommand{\pageHomePostUpload}{\insertPDFfigureManual[page=1,width=\verticalScreenshotWidth][H]{resources/mobile-app/page-home-post-upload.png}{\ifenglish Home Page after uploading \else หน้าแรกหลังจากอัปโหลด \fi}{page-home-post-upload}}

\newcommand{\pageRecordDefault}{\insertPDFfigureManual[page=1,width=\horizontalScreenshotWidth][H]{resources/mobile-app/page-record-default.png}{\ifenglish Record Page before recording \else หน้าบันทึกวิดีโอก่อนการบันทึกวิดีโอ \fi}{page-record-default}} 

\newcommand{\pageRecordRecording}{\insertPDFfigureManual[page=1,width=\horizontalScreenshotWidth][H]{resources/mobile-app/page-record-recording.png}{\ifenglish Record Page during recording \else หน้าบันทึกวิดีโอขณะการบันทึกวิดีโอ \fi}{page-record-recording}} 

\newcommand{\pageRecordPause}{\insertPDFfigureManual[page=1,width=\horizontalScreenshotWidth][H]{resources/mobile-app/page-record-pause.png}{\ifenglish Record Page during pausing \else หน้าบันทึกวิดีโอขณะหยุดชั่วคราว \fi}{page-record-pause}} 

\newcommand{\pageInfoDefault}{\insertPDFfigureManual[page=1,width=\verticalScreenshotWidth][H]{resources/mobile-app/page-info-default.png}{\ifenglish Information Page \else ข้อมูลผู้ใช้ \fi}{page-info-default}} 

\subsection{\ifenglish Mobile Application Interface \else หน้าการใช้งานแอปพลิเคชันมือถือ \fi}
\ifenglish
In this section, we describe the user interface of the mobile application, detailing how users interact with the system through various pages and providing guidelines on using different functions.
\else
ในส่วนนี้อธิบายหน้าตาของแอปพลิเคชันมือถือ โดยระบุวิธีที่ผู้ใช้สื่อสารกับระบบผ่านหน้าต่างต่าง ๆ และให้คำแนะนำในการใช้ฟังก์ชันต่าง ๆ
\fi

\subsubsection{\ifenglish Login Page \else หน้าเข้าสู่ระบบ \fi}
\ifenglish
\textbf{Purpose:} Allows users to log in by providing credentials.\\
\textbf{Usage:} 
\begin{itemize}
    \item Display of input fields for username and password.
    \item Error notifications for invalid credentials.
    \item Navigation to the Home Page upon successful login.
\end{itemize}
\else
\textbf{จุดประสงค์:} ให้ผู้ใช้เข้าสู่ระบบโดยกรอกข้อมูลประจำตัว\\
\textbf{วิธีใช้งาน:} 
\begin{itemize}
    \item แสดงช่องกรอกข้อมูลสำหรับชื่อผู้ใช้และรหัสผ่าน
    \item แจ้งเตือนเมื่อข้อมูลไม่ถูกต้อง
    \item นำทางไปยังหน้าแรกหลังจากเข้าสู่ระบบสำเร็จ
\end{itemize}
\fi

\pageLoginDefault

\ifenglish
    As shown in \ref{fig:page-login-default}, the login page features text input fields for the username and password. After entering their credentials, users can click the login button to proceed.
\else
    ดังที่แสดงใน \ref{fig:page-login-default} หน้าจอเข้าสู่ระบบมีช่องกรอกข้อมูลสำหรับชื่อผู้ใช้และรหัสผ่าน หลังจากกรอกข้อมูลเสร็จสิ้น ผู้ใช้สามารถคลิกปุ่มเข้าสู่ระบบเพื่อดำเนินการต่อ
\fi

\pageLoginException

\ifenglish
    As shown in \ref{fig:page-login-exception}, if the user enters incorrect credentials, an error message will be displayed, prompting the user to try again.
\else
    ดังที่แสดงใน \ref{fig:page-login-exception} หากผู้ใช้กรอกข้อมูลไม่ถูกต้อง จะมีข้อความแสดงข้อผิดพลาดและให้ผู้ใช้ลองอีกครั้ง
\fi

\subsubsection{\ifenglish Home Page \else หน้าแรก \fi}
\ifenglish
\textbf{Purpose:} Lists all stored video sessions and enables users to upload or delete sessions.\\
\textbf{Usage:}
\begin{itemize}
    \item Display of video session thumbnails and metadata.
    \item Options to upload or delete a selected session.
    \item Display upload progress.
\end{itemize}
\else
\textbf{จุดประสงค์:} แสดงรายการเซสชันวิดีโอที่ถูกเก็บไว้ในอุปกรณ์ พร้อมทั้งให้ผู้ใช้ทำการอัปโหลดและลบเซสชัน\\
\textbf{วิธีใช้งาน:}
\begin{itemize}
    \item แสดงภาพย่อและข้อมูลประกอบของเซสชันวิดีโอ
    \item มีตัวเลือกสำหรับอัปโหลดหรือลบเซสชันที่เลือก
    \item แสดงความคืบหน้าของการอัปโหลด
\end{itemize}
\fi

\pageHomeEmpty

\ifenglish
    As shown in \ref{fig:page-home-empty}, when the device doesn't have any video sessions, the page displays a message indicating that there are no sessions available.
\else
    ดังที่แสดงใน \ref{fig:page-home-empty} เมื่ออุปกรณ์ไม่มีเซสชันวิดีโอ หน้าจะแสดงข้อความแสดงว่าไม่มีเซสชันที่ใช้ได้
\fi 

\pageHomePreUpload

\ifenglish
    As shown in \ref{fig:page-home-pre-upload}, if the device has video sessions, the page displays a list of cards, each representing a video session. The card contains a thumbnail, title, and metadata.
\else
    ดังที่แสดงใน \ref{fig:page-home-pre-upload} หากอุปกรณ์มีเซสชันวิดีโอ หน้าจะแสดงรายการของการ์ด ซึ่งแต่ละการ์ดแทนเซสชันวิดีโอและประกอบด้วยภาพย่อ ชื่อ และข้อมูลประกอบ
\fi

\pageHomeExtendedCard

\ifenglish
    As shown in \ref{fig:page-home-extended-card}, when a card is selected, it expands to show additional options, that are uploading and deleting the session.
\else
    ดังที่แสดงใน \ref{fig:page-home-extended-card} เมื่อการ์ดถูกเลือก มันจะขยายเพื่อแสดงตัวเลือกเพิ่มเติม ซึ่งคือการอัปโหลดและลบเซสชัน
\fi

\pageHomeUploading

\ifenglish
    As shown in \ref{fig:page-home-uploading}, when a session is being uploaded, the card displays upload progress.
\else
    ดังที่แสดงใน \ref{fig:page-home-uploading} เมื่อเซสชันกำลังถูกอัปโหลด การ์ดจะแสดงความคืบหน้าของการอัปโหลด
\fi 

\pageHomeUploadingNotificaiton

\ifenglish
    As shown in \ref{fig:page-home-uploading-notification}, the upload process also sends notifications about the upload progress.
\else
    ดังที่แสดงใน \ref{fig:page-home-uploading-notification} กระบวนการอัปโหลดยังส่งการแจ้งเตือนเกี่ยวกับความคืบหน้าของการอัปโหลด
\fi

\pageHomePostUpload

\ifenglish
    As shown in \ref{fig:page-home-post-upload}, after the upload is complete, the card displays a completion message and only the delete option will be shown.
\else
    ดังที่แสดงใน \ref{fig:page-home-post-upload} เมื่อการอัปโหลดเสร็จสิ้น การ์ดจะแสดงข้อความเสร็จสิ้นและจะแสดงเฉพาะตัวเลือกลบเท่านั้น
\fi 

\subsubsection{\ifenglish Recording Page \else หน้าบันทึกวิดีโอ \fi}
\ifenglish
\textbf{Purpose:} Responsible for the recording of video sessions, including starting, pausing, and stopping the recording.\\
\textbf{Usage:}
\begin{itemize}
    \item Button controls for Start Recording, Pause Recording, and Stop Recording.
    \item Display of functions available based on recording status.
\end{itemize}
\else
\textbf{จุดประสงค์:} รับผิดชอบในการบันทึกเซสชันวิดีโอ ได้แก่การเริ่มต้น หยุดชั่วคราว และหยุดการบันทึก\\
\textbf{วิธีใช้งาน:}
\begin{itemize}
    \item ปุ่มควบคุมสำหรับเริ่มบันทึก หยุดชั่วคราว และหยุดบันทึก
    \item แสดงฟังก์ชันที่สามารถทำได้ตามสถานะการบันทึก
\end{itemize}
\fi

\pageRecordDefault

\ifenglish
    As shown in \ref{fig:page-record-default}, on the recording page, users can start recording by clicking the red circular button.
\else
    ดังที่แสดงใน \ref{fig:page-record-default} บนหน้าบันทึกวิดีโอ ผู้ใช้สามารถเริ่มต้นการบันทึกโดยคลิกปุ่มวงกลมสีแดง
\fi

\pageRecordRecording

\ifenglish
    As shown in \ref{fig:page-record-recording}, during recording, users can pause by clicking the pause icon or stop recording by clicking the square icon.
\else
    ดังที่แสดงใน \ref{fig:page-record-recording} ในระหว่างการบันทึก ผู้ใช้สามารถหยุดชั่วคราวโดยคลิกไอคอนหยุดชั่วคราว หรือหยุดการบันทึกโดยคลิกไอคอนรูปสี่เหลี่ยม
\fi

\pageRecordPause

\ifenglish
    As shown in \ref{fig:page-record-pause}, when the recording is paused, users can resume by clicking the red circular button or stop the recording by clicking the square icon.
\else
    ดังที่แสดงใน \ref{fig:page-record-pause} เมื่อการบันทึกหยุดชั่วคราว ผู้ใช้สามารถดำเนินการบันทึกต่อโดยคลิกปุ่มวงกลมสีแดง หรือหยุดการบันทึกโดยคลิกไอคอนรูปสี่เหลี่ยม
\fi

\subsubsection{\ifenglish Information Page \else หน้าข้อมูลผู้ใช้ \fi}
\ifenglish
\textbf{Purpose:} Displays user information and provides a log-out option.\\
\textbf{Usage:}
\begin{itemize}
    \item Display of user profile details.
    \item Option to log out.
\end{itemize}
\else
\textbf{จุดประสงค์:} แสดงข้อมูลผู้ใช้และมีตัวเลือกให้ผู้ใช้ออกจากระบบ\\
\textbf{วิธีใช้งาน:}
\begin{itemize}
    \item แสดงรายละเอียดของโปรไฟล์ผู้ใช้
    \item มีตัวเลือกสำหรับออกจากระบบ
\end{itemize}
\fi

\pageInfoDefault

\ifenglish
    As shown in \ref{fig:page-info-default}, in the information page, the user can log out by clicking the log out button.
\else
    ดังที่แสดงใน \ref{fig:page-info-default} ในหน้าข้อมูลผู้ใช้ ผู้ใช้สามารถออกจากระบบโดยคลิกปุ่มออกจากระบบ
\fi
