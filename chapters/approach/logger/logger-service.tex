\subsection{\ifenglish Service \else บริการ \fi}

\ifenglish
\else
บริการของระบบ Log ใช้สำหรับการจัดการข้อมูล Log ของผู้ใช้งานในระบบ 

ฟังก์ชันใน Log Services ประกอบด้วย
\begin{itemize}
    \item \textbf{getInstance(): LogService}: เป็นฟังก์ชันที่ใช้ในการสร้างหรือเรียกใช้งาน instance ของ \\LogService โดยใช้ Singleton pattern เพื่อให้แน่ใจว่ามีเพียง instance เดียวเท่านั้นที่ถูกสร้างขึ้นในระบบ
    \item \textbf{createLog(logData: ILog): Promise<\{success: boolean, data: ILogDocument\}>}: เป็นฟังก์ชันที่ใช้ในการสร้าง Log ใหม่ลงไปในฐานข้อมูลของระบบ Log
    \item \textbf{getLogs(query: FilterQuery<ILogDocument> = \{\}, options: optionType = \{\}): \\Promise<\{success: boolean, data: log :ILogDocument\}>}: เป็นฟังก์ชันที่ใช้ในการดึงข้อมูล Log ของผู้ใช้งานได้ โดยสามารถกำหนดเงื่อนไขเรื่องของกำหนดของเขตกับการข้ามในการค้นหาข้อมูลได้
    \item \textbf{getUserLogs(userId: string): Promise<\{success: boolean, data: ILogDocument\}>}: เป็นฟังก์ชันที่ใช้ในการดึงข้อมูล Log ของผู้ใช้งานได้ โดยจะส่งเป็นข้อมูลของ 7 วันล่าสุดเท่านั้น
\end{itemize}

ข้อมูลใน LogStatus ประกอบไปด้วย
\begin{lstlisting}
    {
        "SUCCESS": "success",
        "FAILED": "failed"
    }
\end{lstlisting}

ข้อมูลใน ILog ถูกจัดเก็บในรูปแบบดังนี้:
\begin{lstlisting}
    {
        "userId": string,
        "description": string,
        "status": LogStatus,
        "timestamp": Date
    }
\end{lstlisting}

ข้อมูลใน ILogDocument ถูกจัดเก็บในรูปแบบดังนี้:
\begin{lstlisting}
    {
        "userId": string,
        "description": string,
        "status": LogStatus,
        "timestamp": Date,
        "_id": string
    }
\end{lstlisting}
\fi