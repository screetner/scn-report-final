\chapter{\ifproject%
\ifenglish Project Structure and Methodology\else โครงสร้างและขั้นตอนการทำงาน\fi
\else%
\ifenglish Project Structure\else โครงสร้างของโครงงาน\fi
\fi
}

\makeatletter

% \renewcommand\section{\@startsection {section}{1}{\z@}%
%                                    {13.5ex \@plus -1ex \@minus -.2ex}%
%                                    {2.3ex \@plus.2ex}%
%                                    {\normalfont\large\bfseries}}

\makeatother
%\vspace{2ex}
% \titleformat{\section}{\normalfont\bfseries}{\thesection}{1em}{}
% \titlespacing*{\section}{0pt}{10ex}{0pt}

\section{การใช้งานพื้นฐาน}
ในส่วนของโมไบล์แอปพลิเคชัน เป็นเครื่องมือที่จะจำเป็นต้องใช้งานกล้องและบันทึกพิกัดตำแหน่งทาง GPS อยู่ตลอดเวลาเพื่อทำการส่งรูปภาพ 
พร้อมกับพิกัดตำแหน่ง แล้วนำไประมวลผลในเซอร์วิสที่ได้ออกแบบเอาไว้ โดยที่เซอร์วิสดังกล่าวจะทำการประมวลผลรูปภาพเพื่อหาป้ายโฆษณาที่สามารถจัดเก็บภาษีได้ 
และหลังจากนั้นก็จะจัดเก็บลงฐานข้อมูลต่อไป 

ในส่วนของเว็บแอปพลิเคชัน จะเป็นส่วนของการแสดงผลข้อมูลที่ได้บันทึกมาได้ส่วนของโมไบล์แอปพลิเคชัน โดยจะแสดงในรูปแบบของหมุดในแผนที่ 
คล้าย ๆ กับการปักหมุดของ Google map โดยที่ในแต่ละหมุดสามารถกดเพื่อดูรายละเอียดต่าง ๆ ได้ เช่น พิกัดของหมุดนั้น 
และลักษณะรูปป้ายในตำแหน่งนั้นๆที่ได้บันทีกมาจากโมไบล์แอปฯ 

\section{การออกแบบระบบพื้นฐานของโครงงาน}
\subsection{Database Design}
\textbf{ประกอบด้วย 4 ตารางได้แก่}
\begin{enumerate}
  \item User table: เนื่องจากระบบต้องมีการ Authentication เพื่อเข้าใช้งานไม่ว่าจะเป็นทั้งส่วนของ โมไบล์แอปฯ หรือเว็บแอปฯ 
  ดังนั้นตารางนี้จึงจะใช้เก็บข้อมูลพืื้นฐานต่าง ๆ ที่จำเป็นต่อการยืนยันตัวตนทั้งหมด 
  \item Role table: ใช้ในการเก็บบทบาททั้งหมดที่มีของระบบ เช่น ผู้ดูแลระบบ ผู้สำรวจ และอื่น ๆ 
  \item Asset table: ใช้ในการเก็บข้อมูลที่ได้รับมาจาก โมไบล์แอปฯ ไม่ว่าจะเป็นตำแหน่งของรูป ชื่อของรูป และประเภทของ asset ที่ตรวจจับได้ 
  \item Asset type table: ใช้ในการเก็บประเภทของ asset ต่างๆที่ระบบสามารถตรวจจับได้ 
  \item Config table: ใช้เก็บการตั้งค่าพื้นฐานต่างๆเช่น ขอบเขตของแผนที่ 
\end{enumerate}

\begin{figure}[h]
  \begin{center}
  \includegraphics[scale=0.8]{resources/ScreetnerDB.png}
  \end{center}
  \caption[Database Design]{Overall Database Design}
  \label{fig:database}
\end{figure}

\subsection{System Design}

\begin{figure}[h]
  \begin{center}
  \includegraphics[scale=0.4]{resources/SystemDesign.png}
  \end{center}
  \caption[System Design]{Overall System Design}
  \label{fig:system design}
\end{figure}

\begin{figure}[h]
  \begin{center}
  \includegraphics[scale=0.7]{resources/WebsiteFlow.png}
  \end{center}
  \caption[Web Application Flow Diagram]{Web Application Flow Diagram}
  \label{fig:web-app flow design}
\end{figure}

\begin{figure}[h]
  \begin{center}
  \includegraphics[scale=0.5]{resources/MobileAppFlow.png}
  \end{center}
  \caption[Mobile Application Flow Diagram]{Mobile Application Flow Diagram}
  \label{fig:mopile-app flow design}
\end{figure}

\begin{figure}[h]
  \begin{center}
  \includegraphics[scale=0.6]{resources/LoginFlow.png}
  \end{center}
  \caption[Login Flow Diagram]{Login Flow Diagram}
  \label{fig:login flow design}
\end{figure}

\begin{figure}[h]
  \begin{center}
  \includegraphics[scale=0.6]{resources/TransitionFlow.png}
  \end{center}
  \caption[Transition Flow Diagram]{Transition Flow Diagram}
  \label{fig:transition flow design}
\end{figure}
