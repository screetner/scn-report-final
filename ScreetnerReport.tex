\documentclass[semifinal]{cpecmu}

%% This is a sample document demonstrating how to use the CPECMU
%% project template. If you are having trouble, see "cpecmu.pdf" for
%% documentation.

\projectNo{S007-2/67}
\acadyear{2024}

\titleTH{สกรีทเนอร์: ระบบสำรวจถนนสำหรับการจัดการสินทรัพย์เมือง}
\titleEN{Screetner: street scanner system for urban asset management}

\author{นายชาญชล ภานุศุภนิรันดร์}{Charnchol Panusupanirun}{640610626}
\author{นายณัฐพงษ์ เทพพิทักษ์}{Natthaphong Thepphithak}{640610634}
\author{นายธนภัทร สมสิทธิ์}{Thanapat Somsit}{640610639}

\cpeadvisor{santi}
\cpecommittee{karn}
\cpecommittee{navadon}

%% Some possible packages to include:
\usepackage[final]{graphicx} % for including graphics
\usepackage{float} % for controlling figure placement with [H]

%% Add bookmarks and hyperlinks in the document.
\PassOptionsToPackage{hyphens}{url}
\usepackage[colorlinks=true,allcolors=Blue4,citecolor=red,linktoc=all]{hyperref}
\def\UrlLeft#1\UrlRight{$#1$}

%% Needed just by this example, but maybe not by most reports
\usepackage{afterpage} % for outputting
\usepackage{pdflscape} % for landscape figures and tables. 
\usepackage{import}

\usepackage{listings} % For Code block
\usepackage{svg}      % For including svg images

\usepackage{xparse}

\newcommand{\figureSettings}{page=1,width=140mm}
\newcommand{\figureSettingsSmall}{page=1,width=100mm}

\lstset{
    basicstyle=\ttfamily\footnotesize, % Sets the font to typewriter
    breaklines=true,      % Enables automatic line breaking
    columns=fullflexible,  % Ensures proper alignment of text
    showstringspaces=false, % Hides spaces in strings
    inputencoding=utf8,
    literate={-}{{\texttt{-}}}1 {:}{{\texttt{:}}}1 {T}{{\texttt{T}}}1 {.}{{\texttt{.}}}1 Z{{\texttt{Z}}}1
    % Somehow, this allow us to user hyphens in code blocks.
    % So don't delete these 3 lines, please.
}

\setcounter{secnumdepth}{3}
\setcounter{tocdepth}{3}

\newcommand{\insertPDFfigure}[4][page=1,width=140mm]{
  \begin{figure}[ht]
    \begin{center}
      \setlength{\fboxsep}{0pt}  % Space between border and image
      \setlength{\fboxrule}{0.5pt}  % Border thickness
      \fbox{\includegraphics[#1]{#2}}
    \end{center}
    \caption[#3]{#3}
    \label{fig:#4}
  \end{figure}
}

\NewDocumentCommand{\insertPDFfigureManual}{O{page=1,width=140mm} O{ht} m m m}{
  \begin{figure}[#2]
    \begin{center}
      \setlength{\fboxsep}{0pt}  % Space between border and image
      \setlength{\fboxrule}{0.5pt}  % Border thickness
      \fbox{\includegraphics[#1]{#3}}
    \end{center}
    \caption{#4}
    \label{fig:#5}
  \end{figure}
}

\newcommand{\insertPDFfigureLandscape}[3]{
  \begin{figure}[p]
    \begin{center}
      \includegraphics[height=0.9\textheight, keepaspectratio]{#1}
    \end{center}
    \caption[#2]{#2}
    \label{fig:#3}
  \end{figure}
}



%% Some other useful packages. Look these up to find out how to use
%% them.
% \usepackage{natbib}    % for author-year citation styles
% \usepackage{txfonts}
% \usepackage{appendix}  % for appendices on a per-chapter basis
% \usepackage{xtab}      % for tables that go over multiple pages
% \usepackage{subfigure} % for subfigures within a figure
% \usepackage{pstricks,pdftricks} % for access to special PostScript and PDF commands
% \usepackage{nomencl}   % if you have a list of abbreviations

%% if you're having problems with overfull boxes, you may need to increase
%% the tolerance to 9999
% \tolerance=9999

% \bibliographystyle{plain}
% \bibliographystyle{IEEEbib}
\bibliographystyle{IEEEtran}

% \renewcommand{\topfraction}{0.85}
% \renewcommand{\textfraction}{0.1}
% \renewcommand{\floatpagefraction}{0.75}

%% Example for glossary entry
%% Need to use glossary option
%% See glossaries package for complete documentation.
\ifglossary
  \newglossaryentry{lorem ipsum}{
    name=lorem ipsum,
    description={derived from Latin dolorem ipsum, translated as ``pain itself''}
  }
\fi

%% Uncomment this command to preview only specified LaTeX file(s)
%% imported with \include command below.
%% Any other file imported via \include but not specified here will not
%% be previewed.
%% Useful if your report is large, as you might not want to build
%% the entire file when editing a certain part of your report.
% \includeonly{chapters/intro,chapters/background}

\begin{document}
\maketitle
\makesignature

\ifproject
\begin{abstractTH}
    โครงการ Screetner (Street Scanner System for Urban Asset Management) 
    เป็นโครงการที่ถูกพัฒนาเพื่ออำนวยความสะดวกในการบริหารจัดการเกี่ยวกับการจัดเก็บภาษีป้าย 
    ด้วยการใช้เทคโนโลยี Object Detection ในการตรวจจับป้ายที่จัดเก็บภาษีได้ 
    โดยใช้แอปพลิเคชันในโทรศัพท์มือถือในการบันทึกข้อมูลภาพในขณะเดียวกันก็จะมี server 
    ที่คอยประมวลผลรูปภาพนั้น และสุดท้ายก็จะมีเว็บแอปพลิเคชันในการแสดงผลรายงานข้อมูลที่ได้จากการบันทึกจากบนโทรศัพท์มือถือ
\end{abstractTH}

\begin{abstract}
    The Screetner project (Street Scanner System for Urban Asset Management) is 
    a project developed to facilitate the management of taxable billboards utilizing 
    Object Detection technology. This is achieved through the use of a mobile application 
    on handheld devices to capture image data, while simultaneously having a server to 
    process the image data. Lastly, there is a web application to display reports derived 
    from the captured data.
\end{abstract}

\iffalse
\begin{dedication}
This document is dedicated to all Chiang Mai University students.

Dedication page is optional.
\end{dedication}
\fi % \iffalse

\begin{acknowledgments}
    กิตติกรรมประกาศ
    โครงงานนี้จะไม่สามารถสําเร็จลุล่วงได้ หากไม่ได้รับความกรุณาจาก รศ.ดร.สันติ พิทักษ์กิจนุกูร อาจารย์ที่ปรึกษา
    ของโครงงาน ที่ได้สละเวลาให้ความช่วยเหลือทั้งให้คําแนะนํา ให้ความรู้และแนวคิดต่างๆ รวมถึง ผศ.ดร.กานต์ ปทานุคม และ ผศ.ดร.นวดนย์ คุณเลิศกิจ คณะกรรมการสอบโครงงาน ที่ให้คําปรึกษาจนทําให้โครงงานเล่มนี้
    เสร็จสมบูรณ์ได้
\acksign{2025}{3}{22}
\end{acknowledgments}%
\fi % \ifproject

\contentspage

\ifproject
\figurelistpage
% \tablelistpage
\fi % \ifproject

% \abbrlist % this page is optional

% \symlist % this page is optional

% \preface % this section is optional


\pagestyle{empty}\cleardoublepage
\normalspacing \setcounter{page}{1} \pagenumbering{arabic} \pagestyle{cpecmu}

\chapter{\ifenglish Introduction\else บทนำ\fi}

\section{\ifenglish Project rationale\else ที่มาของโครงงาน\fi}

% Okay there is actaully a tab down here vvv
การจัดเก็บภาษีถือเป็นหนึ่งในรายได้หลักของประเทศไม่ว่าจะเป็นภาษีทางตรง อย่างเช่น ภาษีทางตรง ภาษีรายได้บุคคลธรรมดา ซึ่งจะจัดเก็บได้
จากประชาชนผู้มีเงินได้ทั่วไป ภาษีเงินได้นิติบุคคลซึ่งเป็นภาษีที่จัดเก็บได้จากเงินได้ของบริษัทหรือห้างหุ้นส่วนนิติบุคคล และยังมีภาษีทางอ้อม เช่น 
ภาษีมูลค่าเพิ่ม ภาษีธุรกิจเฉพาะ ซึ่งเงินที่ได้จากการเก็บภาษีเหล่าล้วนนำไปให้รัฐบาลใช้ในการพัฒนาประเทศให้เจริญก้าวหน้า  

ภาษีป้ายก็เป็นส่วนหนึ่งของรายได้ท้องถิ่นที่สามารถจัดเก็บได้โดยองค์กรปกครองส่วนท้องถิ่น โดยที่ภาษีลักษณะนี้เมื่อจัดเก็บได้แล้ว ทางท้องถิ่น
ไม่จำเป็นต้องส่งคืนให้ทางรัฐ สามารถนำไปใช้จัดการบริหารพัฒนาภายในท้องถิ่นของตนเองได้ 
แต่ด้วยความสามารถในการจัดเก็บภาษีป้ายขององค์กรปกครองส่วนท้องถิ่นในแต่ละที่ ขึ้นอยู่กับปัจจัยหลาย ๆ อย่าง เช่น 
การที่ไม่สามารถรู้ได้ว่าป้ายที่สามารถจัดเก็บภาษีได้นั้นอยู่ที่ตำแหน่งใดในเขตปกครอง ซึ่งมีส่วนทำให้ประสิทธิภาพในการค้นหาป้ายภายในท้องถิ่นที่มีอยู่ทำได้อยู่จำกัด
และเป็นขั้นตอนที่ต้องใช้กำลังคนในการตรวจสอบเป็นอย่างมาก 
ดังนั้นจากปัญหาในจุดที่กล่าวมาทำให้เกิดโครงงานที่เป็นเครื่องมือที่ช่วยในการตรวจจับหาป้ายที่คาดว่าจะสามารถนำไปจัดเก็บภาษี 
และรายงานผลให้กับแต่ละองค์กรปกครองส่วนท้องถิ่นให้ไปจัดเก็บภาษีจากป้ายเหล่านี้

\section{\ifenglish Objectives\else วัตถุประสงค์ของโครงงาน\fi}
\begin{enumerate}
    \item เพื่อสร้างระบบครบวงจรในการรับวิดีโอแล้วประมวลผลตรวจจับหาป้ายอัตโนมัติ
    \item เพื่อพัฒนาแอปพลิเคชันที่ใช้ในการอัดวิดีโอเพื่อที่จะส่งให้ระบบประมวลผล 
    \item เพื่อพัฒนาเครื่องมือในการรายงานป้ายที่ค้นพบภายในพื่นที่การปกครองส่วนท้องถิ่นสำหรับการไปจัดเก็บภาษี 
\end{enumerate}

\section{\ifenglish Project scope\else ขอบเขตของโครงงาน\fi}

\subsection{\ifenglish Hardware scope\else ขอบเขตด้านฮาร์ดแวร์\fi}

กล้องถ่ายของโทรศัพท์แต่ละเครื่องจะมีคุณภาพและลักษณ์การร่ายที่แตกต่างกัน 
ซึ่งอาจส่งผลต่อการตรวจจับวัตถุทำให้เวลานำรูปภาพที่ได้นำไปประมวลจะได้ผลลัพธ์ที่แตกต่างกัน 
ซึ่งโทรศัพท์ที่ได้ใช้ในการเก็บข้อมูลที่นำไปสร้างโมเดลมีอยู่ด้วยกัน 2 เครื่อง โดยมีคุณภาพของกล้องถ่ายรูปดังนี้
\begin{itemize}
    \item Xiaomi 11T Pro ความละเอียด 108 ล้านพิกเซล 
    \item Samsung Galaxy A50s ความละเอียด 48 ล้านพิกเซล 
\end{itemize}

ความสูงของรถแต่ละคัน และมุมกล้องในการถ่ายภาพมักมีความแตกต่างกันไป ซึ่งอาจส่งผลให้ประสิทธิภาพในการตรวจจับวัตถุได้ไม่เท่ากัน 
โดยรถยนต์ที่ใช้ในการอัดวิดีโอสำหรับในการเทรนโมเดลเป็น Honda City 2024 

Mobile application ที่เป็นส่วนของการส่งข้อมูลภาพไปยังเซิฟเวอร์จำเป็นต้องเชื่อมต่ออินเทอร์เน็ตอยู่ตลอดทั้งการใช้งาน เนื่องจากต้องมีการส่งข้อมูลตลอดเวลา 
ทั้งนี้สืบก็จะมีเรื่องของการใช้งานทรัพยากรแบตเตอร์มากตามไปด้วย และในการของการแสดงผลที่เป็นเว็บแอปพลิเคชันจะสามารถใช้งานได้เฉพาะ
ในคอมพิวเตอร์เท่านั้น 

\subsection{\ifenglish Software scope\else ขอบเขตด้านซอฟต์แวร์\fi}

ในการเก็บภาษีป้ายนั้นจะถูกแบ่งออกเป็นป้ายหลาย ๆ ประเภท อย่างเช่น ป้ายที่มีอักษรไทยล้วน ป้ายที่มีอักษรไทยปนกับอักษรต่างประเทศหรือปนกับภาพ 
และหรือเครื่องหมาย ป้ายที่ไม่มีอักษรไทย ไม่ว่าจะมีภาพและหรือเครื่องหมายใด ๆ ซึ่งแต่ละประเภทนั้นจะมีอัตราการเก็บภาษีที่แตกต่างกันออกไป 
แต่ในการประมวลผลในเซิฟเวอร์นั้นจะไม่มีการตรวจสอบและแบ่งแยกประเภทของป้าย และจะรวบรวมเป็นคลาสประเภทเดียวกันแทน อีกทั้งป้ายที่สามารถจัดเก็บภาษีได้บางประเภทมีลักษณะคล้ายกับป้ายบอกทางและป้ายจราจรจึงอาจทำให้มีข้อผิดพลาดเกิดขึ้นในการตรวจจับในบางสถานการณ์

\section{\ifenglish Expected outcomes\else ประโยชน์ที่ได้รับ\fi}
\begin{enumerate}
    \item ได้เครื่องมือที่ช่วยอำนวยความสะดวกในการเก็บภาษีป้ายให้มีประสิทธิภาพมากยิ่งขึ้น
\end{enumerate}

\section{\ifenglish Technology and tools\else เทคโนโลยีและเครื่องมือที่ใช้\fi}

% \subsection{\ifenglish Hardware technology\else เทคโนโลยีด้านฮาร์ดแวร์\fi}

\subsection{\ifenglish Software technology\else เทคโนโลยีด้านซอฟต์แวร์\fi}

\begin{enumerate}

    \item JetBrainIDEs เป็นชุดเครื่องมือพัฒนาโปรแกรมจาก JetBrains ที่ประกอบด้วย IDEs หลายตัว เช่น IntelliJ IDEA, PyCharm, และ WebStorm ซึ่งช่วยในการพัฒนาโปรแกรมในภาษาต่าง ๆ อย่างมีประสิทธิภาพ

    \item Data Grip เป็นเครื่องมือจัดการฐานข้อมูลจาก JetBrains ที่ช่วยในการเชื่อมต่อและจัดการฐานข้อมูลหลายประเภท เช่น MySQL, PostgreSQL, และ SQLite ซึ่งช่วยให้นักพัฒนาสามารถทำงานกับฐานข้อมูลได้ง่ายขึ้น

    \item Python เป็นภาษาโปรแกรมมิ่งที่มีความยืดหยุ่นสูงและสามารถนำมาใช้ในการพัฒนาโปรแกรมต่าง ๆ ได้หลากหลาย ซึ่งมีความเหมาะสมในการ
    ใช้งานในโครงการที่ต้องการประมวลผลข้อมูลที่ซับซ้อนและมีขนาดใหญ่ อย่างเช่น โมเดลการเรียนรู้เชิงลึก ที่พวกเราจะนำไปใช้กับการตรวจจับวัตถุ
    
    \item Typescript คือภาษาคอมพิวเตอร์ที่ใช้ในการพัฒนาเว็บร่วมกับ HTML เพื่อให้เว็บมีลักษณะแบบไดนามิก หมายถึง เว็บสามารถตอบสนองกับ
    ผู้ใช้งานหรือแสดงเนื้อหาที่แตกต่างกันไปโดยจะอ้างอิงตาม เว็บบราวเซอร์ที่ผู้เข้าชมเว็บใช้งานอยู่ 

    \item Golang เป็นภาษาการเขียนโปรแกรมที่พัฒนาโดย Google ซึ่งมีประสิทธิภาพสูงและเหมาะสำหรับการพัฒนาแอปพลิเคชันที่ต้องการความเร็วและความเสถียร

    \item Tusd เป็นเซิร์ฟเวอร์ที่ใช้ในการอัปโหลดไฟล์ขนาดใหญ่แบบต่อเนื่อง (resumable file uploads) ซึ่งช่วยให้การอัปโหลดไฟล์มีความเสถียรและไม่ขาดตอน

    \item Azure Logic Apps เป็นบริการของ Microsoft Azure ที่ช่วยในการสร้างและจัดการเวิร์กโฟลว์อัตโนมัติสำหรับการรวมระบบและการประมวลผลข้อมูล

    \item Azure Blob Storage เป็นบริการจัดเก็บข้อมูลแบบออบเจ็กต์ของ Microsoft Azure ที่ใช้ในการจัดเก็บข้อมูลขนาดใหญ่ เช่น ไฟล์วิดีโอและรูปภาพ

    \item Azure App Instance เป็นบริการของ Microsoft Azure ที่ใช้ในการโฮสต์และจัดการแอปพลิเคชันบนคลาวด์

    \item Azure Container Registry เป็นบริการของ Microsoft Azure ที่ใช้ในการจัดเก็บ จัดการ และเรียกใช้งานคอนเทนเนอร์

    \item Azure Log Analytics workspace เป็นบริการของ Microsoft Azure ที่ใช้ในการจัดการ จัดเก็บ และวิเคราะห์ข้อมูลของระบบเช่น Log และ Metric

    \item Azure Email Communication Service เป็นบริการของ Microsoft Azure ที่ใช้ในการส่งอีเมลและการสื่อสารอื่น ๆ ระหว่างระบบ

    \item Flutter เป็นเฟรมเวิร์กที่พัฒนาโดย Google ที่ใช้ในการพัฒนาแอปพลิเคชันข้ามแพลตฟอร์ม (cross-platform) ทั้งบน iOS และ Android ด้วยโค้ดเบสเดียว
    
    \item Next.js เป็นเฟรมเวิร์กที่ใช้ในการพัฒนาเว็บแอปพลิเคชันแบบเซิร์ฟเวอร์ไซด์เรนเดอริ่ง (SSR) และสเตติกไซต์เจเนอเรชัน (SSG) 
    ซึ่งช่วยให้การพัฒนาเว็บมีประสิทธิภาพและความเร็วสูงขึ้น และยังมีฟีเจอร์ที่ช่วยในการทำ SEO ได้ดีขึ้น
    
    \item YOLOv8 เป็นระบบที่ใช้ในการพัฒนาโนโมเดลตรวจจับวัตถุความเร็วสูงแบบเวลาจริง ด้วยการเรียนรู้เชิงลึกและการมองเห็นคอมพิวเตอร์ 
    
    \item Figma เครื่องมือออกแบบเว็บไซต์ แอปพลิเคชัน โลโก้ และอื่น ๆ ทําให้นักออกแบบ UX/UI สะดวก มากขึ้น ผ่านการใช้ฟีเจอร์ต่าง ๆ 
    ซึ่งมีจุดเด่นอยู่ที่การใช้งานบนได้ทุกระบบปฏิบัติการ และยังมี Community ที่ผู้ใช้สามารถแชร์ไฟล์งาน Prototype หรือ Plug-in ต่าง ๆ 
    แล้วนําไปปรับใช้กับงานของตัว เองได้ 

    \item Linux เป็นระบบปฏิบัติการ (Operating System) ที่เป็น Open Source และเป็นพื้นฐานบนหลักการของ Unix ซึ่งถูกพัฒนาขึ้นโดย 
    Linus Torvalds ในปี ค.ศ. 1991 ซึ่งเป็นระบบปฎิบัตการที่เราจะนำมาใช้งาน 

    \item Kong เป็น API Gateway ที่ช่วยในการจัดการ API และการเชื่อมต่อระหว่างบริการต่าง ๆ ในระบบ ซึ่งช่วยเพิ่มความปลอดภัยและประสิทธิภาพในการทำงานของ API

    \item Docker เป็นแพลตฟอร์มที่ใช้ในการสร้าง จัดส่ง และรันแอปพลิเคชันในคอนเทนเนอร์ ซึ่งช่วยให้การพัฒนาและการนำแอปพลิเคชันไปใช้งานมีความยืดหยุ่นและรวดเร็ว

    \item Github Action เป็นเครื่องมือที่ใช้ในการทำ CI/CD (Continuous Integration/Continuous Deployment) บนแพลตฟอร์ม GitHub ซึ่งช่วยให้การทดสอบและการนำโค้ดไปใช้งานเป็นไปอย่างอัตโนมัติและมีประสิทธิภาพ

    \item Draw.io เป็นเครื่องมือออนไลน์ที่ใช้ในการสร้างไดอะแกรมและแผนภาพต่าง ๆ เช่น แผนภาพการไหล (flowchart) และแผนภาพสถาปัตยกรรมระบบ ซึ่งช่วยให้การออกแบบและสื่อสารข้อมูลเป็นไปอย่างมีประสิทธิภาพ

    \item Postman เป็นเครื่องมือที่ใช้ในการทดสอบ API ซึ่งช่วยให้นักพัฒนาสามารถส่งคำขอ (request) และดูผลลัพธ์ (response) ของ API ได้อย่างง่ายดาย

    \item PostgreSQL เป็นระบบจัดการฐานข้อมูลเชิงสัมพันธ์ (RDBMS) ที่มีความเสถียรและมีประสิทธิภาพสูง ซึ่งใช้ในการจัดการและเก็บข้อมูลในโครงการ

    \item MongoDB เป็นระบบจัดการฐานข้อมูลแบบ NoSQL ที่มีความยืดหยุ่นสูงและสามารถจัดการข้อมูลที่ไม่มีโครงสร้าง (unstructured data) ได้อย่างมีประสิทธิภาพ

    \item Redis เป็นฐานข้อมูลแบบ key-value ที่ทำงานในหน่วยความจำ (in-memory) ซึ่งมีความเร็วสูงและเหมาะสำหรับการจัดเก็บข้อมูลที่ต้องการการเข้าถึงอย่างรวดเร็ว

    \item Roboflow เป็นเครื่องมือที่สามารถใช้ทำการ Labeling ข้อมูล และสร้าง Dataset สำหรับการเทรนโมเดล Computer Vision ได้อย่างง่ายดาย
\end{enumerate}

\section{\ifenglish Project plan\else แผนการดำเนินงาน\fi}

\begin{plan}{10}{2024}{3}{2025}
    \planitem{10}{2024}{11}{2024}{ทดลองพัฒนา backend core logic ด้วย python}
    \planitem{11}{2024}{2}{2025}{เปลี่ยนมาพัฒนา backend core logic ด้วย typescript โดยใช้งาน elysia framework}
    \planitem{10}{2024}{3}{2025}{พัฒนา frontend ด้วย typescript}
    \planitem{10}{2024}{3}{2025}{พัฒนาแอปพลิเคชันด้วย flutter}
    \planitem{10}{2024}{12}{2024}{พัฒนาระบบ cloud service ด้วย azure ที่ใช้สำหรับประมวลผล}
    \planitem{10}{2024}{2}{2025}{พัฒนาโมเดลการเรียนรู้เชิงลึกด้วย YOLOv8}
    \planitem{10}{2024}{3}{2025}{พัฒนาระบบตรวจจับป้ายด้วยโมเดลที่พัฒนามาข้างต้น}
    \planitem{1}{2024}{2}{2025}{ทดสอบระบบและปรับปรุงแก้ไข}
    \planitem{2}{2025}{3}{2025}{ทำรายงาน คลิปวิดีโอ และโปสเตอร์}
    \planitem{3}{2025}{3}{2025}{นำเสนอผลงาน}
\end{plan}

% \section{\ifenglish Roles and responsibilities\else บทบาทและความรับผิดชอบ\fi}
% \begin{itemize}
%     \item นาย ชาญชล ภานุศุภนิรันดร์: ทำหน้าที่ศึกษาค้นคว้าข้อมูลที่จะนํามาใช้ในโครงงาน และจัดการเก็บข้อมูลและเชื่อม ส่วนต่อประสานเชิงประยุกต์ 
%     (API: Application Programming Interface) 
%     \item นาย ณัฐพงษ์ เทพพิทักษ์: ทําหน้าที่ศึกษาค้นคว้าข้อมูลที่จะนํามาใช้ในโครงงาน และพัฒนาโมไบล์แอปพลิเคชันถ่ายวิดีโอสำหรับการตรวจจับวัตถุ 
%     \item นาย ธนภัทร สมสิทธิ์: ทําหน้าที่ศึกษาค้นคว้าข้อมูลที่จะนํามาใช้ในโครงงาน และพัฒนาเว็บแอปพลิเคชันรายงานผลข้อมูลหลังจากการประมวลผลตรวจจับข้อมูล 
% \end{itemize}

\section{\ifenglish%
Impacts of this project on society, health, safety, legal, and cultural issues
\else%
ผลกระทบด้านสังคม สุขภาพ ความปลอดภัย กฎหมาย และวัฒนธรรม
\fi}

การพัฒนาระบบในการตรวจจับป้ายที่สามารถนำไปเก็บภาษีได้นั้น จะช่วยอำนวยความสะดวกให้สามารถจัดการได้ง่ายและสะดวกยิ่งขึ้น 
ซึ่งมีผลกระทบในด้านกฏหมายเพราะภาษีป้ายเป็นภาษีที่จัดเก็บจากป้ายที่ แสดงชื่อ ยี่ห้อ หรือเครื่องหมายที่ใช้ในการประกอบ การค้า หรือประกอบกิจการอื่นเพื่อหารายได้ หรือ 
โฆษณาการค้า ซึ่งในส่วนของการเสียนั้นก็ขึ้นอยู่กับประเภทของป้ายตามที่กฏหมายกำหนด และรายได้ที่ได้จากการจัดเก็บภาษีก็จะถูกนำไปพัฒนาบ้านเมืองต่อไป

\include{chapters/background}
\include{chapters/approach}
\include{chapters/eval}
% \ifproject
% \chapter{\ifenglish Conclusions and Discussions\else บทสรุปและข้อเสนอแนะ\fi}

\section{\ifenglish Conclusions\else สรุปผล\fi}

โครงงานนี้ประสบความสำเร็จในการพัฒนาระบบครบทั้ง 3 ส่วนตามวัตถุประสงค์ที่กำหนดไว้ ประกอบด้วย (1) ส่วน web front-end สำหรับผู้ใช้งานทั่วไปและผู้ดูแลระบบ ที่ช่วยให้การเข้าถึงและจัดการข้อมูลเป็นไปอย่างมีประสิทธิภาพ (2) ส่วน mobile application สำหรับการตรวจจับและประมวลผลป้ายจราจรแบบเรียลไทม์ และ (3) ส่วน back-end ที่รองรับการประมวลผลข้อมูล การจัดเก็บ และการเชื่อมต่อระหว่างส่วนต่างๆ ของระบบ แม้ว่าระบบจะสามารถทำงานได้ตามเป้าหมายหลัก แต่ยังพบข้อจำกัดบางประการที่สำคัญ ดังนี้

\begin{itemize}
    \item ความยาววีดีโอต่อ 1 เซสชั่นไม่ควรเกิน 30 นาที
    \item ระบบตรวจจับป้ายอาจมีการทำงานผิดพลาด เนื่องจากความแม่นยำของตัวโมเดลไม่ได้สูงมาก
\end{itemize}

\section{\ifenglish Challenges\else ปัญหาที่พบและแนวทางการแก้ไข\fi}

\begin{itemize}
    \item การทำงานของระบบตรวจจับป้ายยังมีความแม่นยำที่ต่ำ เนื่องจากการเตรียมข้อมูลสำหรับการเรียนรู้ของโมเดลไม่เพียงพอ ดังนั้นวิธีการแก้ไขคือการเพิ่มข้อมูลสำหรับการเรียนรู้ของโมเดลและปรับปรุงโมเดลให้มีความแม่นยำมากขึ้น
    \item ความยาววิดีโอที่ไม่ควรเกิน 30 นาที นั้นสืบเนื่องมาจากระบบ auto scale ที่ได้ออกแบบมานั้นมีระบนยะเวลาการทำงานต่อ 1 เซสชั่นประมาณ 1 ชั่วโมง ซึ่งควารมยาวของการประมวลผลจะใช้เวลาเป็น 2 เท่าของความยาววิดีโอ ดังนั้นความยาววิดีโอจึงไม่ควรที่จะเกิน 30 นาที และวิธีการแก้ไขคือการปรับปรุงระบบ auto scale ให้สามารถทำงานได้เร็วขึ้น
\end{itemize}

\section{\ifenglish%
Suggestions and further improvements
\else%
ข้อเสนอแนะและแนวทางการพัฒนาต่อ
\fi
}

ข้อเสนอแนะเพื่อพัฒนาโครงงานนี้ต่อไป มีดังนี้
% \fi

\bibliography{ScreetnerReport}

% \ifproject
% \normalspacing
% \appendix
% \include{chapters/appendix}

%% Display glossary (optional) -- need glossary option.
% \ifglossary\glossarypage\fi

%% Display index (optional) -- need idx option.
% \ifindex\indexpage\fi

% \begin{biosketch}
% \begin{center}
%   \includegraphics[width=1.5in]{mugshot.jpg}
% \end{center}
% Your biosketch goes here. Make sure it sits inside
% the \texttt{biosketch} environment.
% \end{biosketch}
% \fi % \ifproject
\end{document}
