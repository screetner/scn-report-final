\documentclass[semifinal]{cpecmu}

%% This is a sample document demonstrating how to use the CPECMU
%% project template. If you are having trouble, see "cpecmu.pdf" for
%% documentation.

\projectNo{S007-2/67}
\acadyear{2024}

\titleTH{สกรีทเนอร์: ระบบสำรวจถนนสำหรับการจัดการสินทรัพย์เมือง}
\titleEN{Screetner: street scanner system for urban asset management}

\author{นายชาญชล ภานุศุภนิรันดร์}{Charnchol Panusupanirun}{640610626}
\author{นายณัฐพงษ์ เทพพิทักษ์}{Natthaphong Thepphithak}{640610634}
\author{นายธนภัทร สมสิทธิ์}{Thanapat Somsit}{640610639}

\cpeadvisor{santi}
\cpecommittee{karn}
\cpecommittee{navadon}

%% Some possible packages to include:
\usepackage[final]{graphicx} % for including graphics

%% Add bookmarks and hyperlinks in the document.
\PassOptionsToPackage{hyphens}{url}
\usepackage[colorlinks=true,allcolors=Blue4,citecolor=red,linktoc=all]{hyperref}
\def\UrlLeft#1\UrlRight{$#1$}

%% Needed just by this example, but maybe not by most reports
\usepackage{afterpage} % for outputting
\usepackage{pdflscape} % for landscape figures and tables. 
\usepackage{import}

\usepackage{listings} % For Code block
\usepackage{svg}      % For including svg images

\newcommand{\figureSettings}{page=1,width=140mm}
\newcommand{\figureSettingsSmall}{page=1,width=100mm}

\lstset{
    basicstyle=\ttfamily\footnotesize, % Sets the font to typewriter
    breaklines=true,      % Enables automatic line breaking
    columns=fullflexible,  % Ensures proper alignment of text
    showstringspaces=false, % Hides spaces in strings
    inputencoding=utf8,
    literate={-}{{\texttt{-}}}1 {:}{{\texttt{:}}}1 {T}{{\texttt{T}}}1 {.}{{\texttt{.}}}1 Z{{\texttt{Z}}}1
    % Somehow, this allow us to user hyphens in code blocks.
    % So don't delete these 3 lines, please.
}

\newcommand{\insertPDFfigure}[4][page=1,width=140mm]{
  \begin{figure}[ht]
    \begin{center}
      \includegraphics[#1]{#2}
    \end{center}
    \caption[#3]{#3}
    \label{fig:#4}
  \end{figure}
}

\newcommand{\insertPDFfigureLandscape}[3]{
  \begin{figure}[p]
    \begin{center}
      \includegraphics[height=0.9\textheight, keepaspectratio]{#1}
    \end{center}
    \caption[#2]{#2}
    \label{fig:#3}
  \end{figure}
}



%% Some other useful packages. Look these up to find out how to use
%% them.
% \usepackage{natbib}    % for author-year citation styles
% \usepackage{txfonts}
% \usepackage{appendix}  % for appendices on a per-chapter basis
% \usepackage{xtab}      % for tables that go over multiple pages
% \usepackage{subfigure} % for subfigures within a figure
% \usepackage{pstricks,pdftricks} % for access to special PostScript and PDF commands
% \usepackage{nomencl}   % if you have a list of abbreviations

%% if you're having problems with overfull boxes, you may need to increase
%% the tolerance to 9999
% \tolerance=9999

% \bibliographystyle{plain}
% \bibliographystyle{IEEEbib}
\bibliographystyle{IEEEtran}

% \renewcommand{\topfraction}{0.85}
% \renewcommand{\textfraction}{0.1}
% \renewcommand{\floatpagefraction}{0.75}

%% Example for glossary entry
%% Need to use glossary option
%% See glossaries package for complete documentation.
\ifglossary
  \newglossaryentry{lorem ipsum}{
    name=lorem ipsum,
    description={derived from Latin dolorem ipsum, translated as ``pain itself''}
  }
\fi

%% Uncomment this command to preview only specified LaTeX file(s)
%% imported with \include command below.
%% Any other file imported via \include but not specified here will not
%% be previewed.
%% Useful if your report is large, as you might not want to build
%% the entire file when editing a certain part of your report.
% \includeonly{chapters/intro,chapters/background}

\begin{document}
\maketitle
\makesignature

\ifproject
\begin{abstractTH}
    โครงการ Screetner (Street Scanner System for Urban Asset Management) 
    เป็นโครงการที่ถูกพัฒนาเพื่ออำนวยความสะดวกในการบริหารจัดการเกี่ยวกับการจัดเก็บภาษีป้าย 
    ด้วยการใช้เทคโนโลยี Object Detection ในการตรวจจับป้ายที่จัดเก็บภาษีได้ 
    โดยใช้แอปพลิเคชันในโทรศัพท์มือถือในการบันทึกข้อมูลภาพในขณะเดียวกันก็จะมี server 
    ที่คอยประมวลผลรูปภาพนั้น และสุดท้ายก็จะมีเว็บแอปพลิเคชันในการแสดงผลรายงานข้อมูลที่ได้จากการบันทึกจากบนโทรศัพท์มือถือ
\end{abstractTH}

\begin{abstract}
    The Screetner project (Street Scanner System for Urban Asset Management) is 
    a project developed to facilitate the management of taxable billboards utilizing 
    Object Detection technology. This is achieved through the use of a mobile application 
    on handheld devices to capture image data, while simultaneously having a server to 
    process the image data. Lastly, there is a web application to display reports derived 
    from the captured data.
\end{abstract}

\iffalse
\begin{dedication}
This document is dedicated to all Chiang Mai University students.

Dedication page is optional.
\end{dedication}
\fi % \iffalse

% \begin{acknowledgments}
% Your acknowledgments go here. Make sure it sits inside the
% \texttt{acknowledgment} environment.

% \acksign{2024}{2}{24}
% \end{acknowledgments}%
\fi % \ifproject

\contentspage

\ifproject
\figurelistpage
% \tablelistpage
\fi % \ifproject

% \abbrlist % this page is optional

% \symlist % this page is optional

% \preface % this section is optional


\pagestyle{empty}\cleardoublepage
\normalspacing \setcounter{page}{1} \pagenumbering{arabic} \pagestyle{cpecmu}

\chapter{\ifenglish Introduction\else บทนำ\fi}

\section{\ifenglish Project rationale\else ที่มาของโครงงาน\fi}

% Okay there is actaully a tab down here vvv
การจัดเก็บภาษีถือเป็นหนึ่งในรายได้หลักของประเทศไม่ว่าจะเป็นภาษีทางตรง อย่างเช่น ภาษีทางตรง ภาษีรายได้บุคคลธรรมดา ซึ่งจะจัดเก็บได้
จากประชาชนผู้มีเงินได้ทั่วไป ภาษีเงินได้นิติบุคคลซึ่งเป็นภาษีที่จัดเก็บได้จากเงินได้ของบริษัทหรือห้างหุ้นส่วนนิติบุคคล และยังมีภาษีทางอ้อม เช่น 
ภาษีมูลค่าเพิ่ม ภาษีธุรกิจเฉพาะ ซึ่งเงินที่ได้จากการเก็บภาษีเหล่าล้วนนำไปให้รัฐบาลใช้ในการพัฒนาประเทศให้เจริญก้าวหน้า  

ภาษีป้ายก็เป็นส่วนหนึ่งของรายได้ท้องถิ่นที่สามารถจัดเก็บได้โดยองค์กรปกครองส่วนท้องถิ่น โดยที่ภาษีลักษณะนี้เมื่อจัดเก็บได้แล้ว ทางท้องถิ่น
ไม่จำเป็นต้องส่งคืนให้ทางรัฐ สามารถนำไปใช้จัดการบริหารพัฒนาภายในท้องถิ่นของตนเองได้ 
แต่ด้วยความสามารถในการจัดเก็บภาษีป้ายขององค์กรปกครองส่วนท้องถิ่นในแต่ละที่ ขึ้นอยู่กับปัจจัยหลาย ๆ อย่าง เช่น 
การที่ไม่สามารถรู้ได้ว่าป้ายที่สามารถจัดเก็บภาษีได้นั้นอยู่ที่ตำแหน่งใดในเขตปกครอง ซึ่งมีส่วนทำให้ประสิทธิภาพในการค้นหาป้ายภายในท้องถิ่นที่มีอยู่ทำได้อยู่จำกัด
และเป็นขั้นตอนที่ต้องใช้กำลังคนในการตรวจสอบเป็นอย่างมาก 
ดังนั้นจากปัญหาในจุดที่กล่าวมาทำให้เกิดโครงงานที่เป็นเครื่องมือที่ช่วยในการตรวจจับหาป้ายที่คาดว่าจะสามารถนำไปจัดเก็บภาษี 
และรายงานผลให้กับแต่ละองค์กรปกครองส่วนท้องถิ่นให้ไปจัดเก็บภาษีจากป้ายเหล่านี้

\section{\ifenglish Objectives\else วัตถุประสงค์ของโครงงาน\fi}
\begin{enumerate}
    \item เพื่อสร้างระบบครบวงจรในการรับวิดีโอแล้วประมวลผลตรวจจับหาป้ายอัตโนมัติ
    \item เพื่อพัฒนาแอปพลิเคชันที่ใช้ในการอัดวิดีโอเพื่อที่จะส่งให้ระบบประมวลผล 
    \item เพื่อพัฒนาเครื่องมือในการรายงานป้ายที่ค้นพบภายในพื่นที่การปกครองส่วนท้องถิ่นสำหรับการไปจัดเก็บภาษี 
\end{enumerate}

\section{\ifenglish Project scope\else ขอบเขตของโครงงาน\fi}

\subsection{\ifenglish Hardware scope\else ขอบเขตด้านฮาร์ดแวร์\fi}

กล้องถ่ายของโทรศัพท์แต่ละเครื่องจะมีคุณภาพและลักษณ์การร่ายที่แตกต่างกัน 
ซึ่งอาจส่งผลต่อการตรวจจับวัตถุทำให้เวลานำรูปภาพที่ได้นำไปประมวลจะได้ผลลัพธ์ที่แตกต่างกัน 
ซึ่งโทรศัพท์ที่ได้ใช้ในการเก็บข้อมูลที่นำไปสร้างโมเดลมีอยู่ด้วยกัน 2 เครื่อง โดยมีคุณภาพของกล้องถ่ายรูปดังนี้
\begin{itemize}
    \item Xiaomi 11T Pro ความละเอียด 108 ล้านพิกเซล 
    \item Samsung Galaxy A50s ความละเอียด 48 ล้านพิกเซล 
\end{itemize}

ความสูงของรถแต่ละคัน และมุมกล้องในการถ่ายภาพมักมีความแตกต่างกันไป ซึ่งอาจส่งผลให้ประสิทธิภาพในการตรวจจับวัตถุได้ไม่เท่ากัน 
โดยรถยนต์ที่ใช้ในการอัดวิดีโอสำหรับในการเทรนโมเดลเป็น Honda City 2024 

Mobile application ที่เป็นส่วนของการส่งข้อมูลภาพไปยังเซิฟเวอร์จำเป็นต้องเชื่อมต่ออินเทอร์เน็ตอยู่ตลอดทั้งการใช้งาน เนื่องจากต้องมีการส่งข้อมูลตลอดเวลา 
ทั้งนี้สืบก็จะมีเรื่องของการใช้งานทรัพยากรแบตเตอร์มากตามไปด้วย และในการของการแสดงผลที่เป็นเว็บแอปพลิเคชันจะสามารถใช้งานได้เฉพาะ
ในคอมพิวเตอร์เท่านั้น 

\subsection{\ifenglish Software scope\else ขอบเขตด้านซอฟต์แวร์\fi}

ในการเก็บภาษีป้ายนั้นจะถูกแบ่งออกเป็นป้ายหลาย ๆ ประเภท อย่างเช่น ป้ายที่มีอักษรไทยล้วน ป้ายที่มีอักษรไทยปนกับอักษรต่างประเทศหรือปนกับภาพ 
และหรือเครื่องหมาย ป้ายที่ไม่มีอักษรไทย ไม่ว่าจะมีภาพและหรือเครื่องหมายใด ๆ ซึ่งแต่ละประเภทนั้นจะมีอัตราการเก็บภาษีที่แตกต่างกันออกไป 
แต่ในการประมวลผลในเซิฟเวอร์นั้นจะไม่มีการตรวจสอบและแบ่งแยกประเภทของป้าย และจะรวบรวมเป็นคลาสประเภทเดียวกันแทน อีกทั้งป้ายที่สามารถจัดเก็บภาษีได้บางประเภทมีลักษณะคล้ายกับป้ายบอกทางและป้ายจราจรจึงอาจทำให้มีข้อผิดพลาดเกิดขึ้นในการตรวจจับในบางสถานการณ์

\section{\ifenglish Expected outcomes\else ประโยชน์ที่ได้รับ\fi}
\begin{enumerate}
    \item ได้เครื่องมือที่ช่วยอำนวยความสะดวกในการเก็บภาษีป้ายให้มีประสิทธิภาพมากยิ่งขึ้น
\end{enumerate}

\section{\ifenglish Technology and tools\else เทคโนโลยีและเครื่องมือที่ใช้\fi}

% \subsection{\ifenglish Hardware technology\else เทคโนโลยีด้านฮาร์ดแวร์\fi}

\subsection{\ifenglish Software technology\else เทคโนโลยีด้านซอฟต์แวร์\fi}

\begin{enumerate}

    \item JetBrainIDEs เป็นชุดเครื่องมือพัฒนาโปรแกรมจาก JetBrains ที่ประกอบด้วย IDEs หลายตัว เช่น IntelliJ IDEA, PyCharm, และ WebStorm ซึ่งช่วยในการพัฒนาโปรแกรมในภาษาต่าง ๆ อย่างมีประสิทธิภาพ

    \item Data Grip เป็นเครื่องมือจัดการฐานข้อมูลจาก JetBrains ที่ช่วยในการเชื่อมต่อและจัดการฐานข้อมูลหลายประเภท เช่น MySQL, PostgreSQL, และ SQLite ซึ่งช่วยให้นักพัฒนาสามารถทำงานกับฐานข้อมูลได้ง่ายขึ้น

    \item Python เป็นภาษาโปรแกรมมิ่งที่มีความยืดหยุ่นสูงและสามารถนำมาใช้ในการพัฒนาโปรแกรมต่าง ๆ ได้หลากหลาย ซึ่งมีความเหมาะสมในการ
    ใช้งานในโครงการที่ต้องการประมวลผลข้อมูลที่ซับซ้อนและมีขนาดใหญ่ อย่างเช่น โมเดลการเรียนรู้เชิงลึก ที่พวกเราจะนำไปใช้กับการตรวจจับวัตถุ
    
    \item Typescript คือภาษาคอมพิวเตอร์ที่ใช้ในการพัฒนาเว็บร่วมกับ HTML เพื่อให้เว็บมีลักษณะแบบไดนามิก หมายถึง เว็บสามารถตอบสนองกับ
    ผู้ใช้งานหรือแสดงเนื้อหาที่แตกต่างกันไปโดยจะอ้างอิงตาม เว็บบราวเซอร์ที่ผู้เข้าชมเว็บใช้งานอยู่ 

    \item Golang เป็นภาษาการเขียนโปรแกรมที่พัฒนาโดย Google ซึ่งมีประสิทธิภาพสูงและเหมาะสำหรับการพัฒนาแอปพลิเคชันที่ต้องการความเร็วและความเสถียร

    \item Tusd เป็นเซิร์ฟเวอร์ที่ใช้ในการอัปโหลดไฟล์ขนาดใหญ่แบบต่อเนื่อง (resumable file uploads) ซึ่งช่วยให้การอัปโหลดไฟล์มีความเสถียรและไม่ขาดตอน

    \item Azure Logic Apps เป็นบริการของ Microsoft Azure ที่ช่วยในการสร้างและจัดการเวิร์กโฟลว์อัตโนมัติสำหรับการรวมระบบและการประมวลผลข้อมูล

    \item Azure Blob Storage เป็นบริการจัดเก็บข้อมูลแบบออบเจ็กต์ของ Microsoft Azure ที่ใช้ในการจัดเก็บข้อมูลขนาดใหญ่ เช่น ไฟล์วิดีโอและรูปภาพ

    \item Azure App Instance เป็นบริการของ Microsoft Azure ที่ใช้ในการโฮสต์และจัดการแอปพลิเคชันบนคลาวด์

    \item Azure Container Registry เป็นบริการของ Microsoft Azure ที่ใช้ในการจัดเก็บ จัดการ และเรียกใช้งานคอนเทนเนอร์

    \item Azure Log Analytics workspace เป็นบริการของ Microsoft Azure ที่ใช้ในการจัดการ จัดเก็บ และวิเคราะห์ข้อมูลของระบบเช่น Log และ Metric

    \item Azure Email Communication Service เป็นบริการของ Microsoft Azure ที่ใช้ในการส่งอีเมลและการสื่อสารอื่น ๆ ระหว่างระบบ

    \item Flutter เป็นเฟรมเวิร์กที่พัฒนาโดย Google ที่ใช้ในการพัฒนาแอปพลิเคชันข้ามแพลตฟอร์ม (cross-platform) ทั้งบน iOS และ Android ด้วยโค้ดเบสเดียว
    
    \item Next.js เป็นเฟรมเวิร์กที่ใช้ในการพัฒนาเว็บแอปพลิเคชันแบบเซิร์ฟเวอร์ไซด์เรนเดอริ่ง (SSR) และสเตติกไซต์เจเนอเรชัน (SSG) 
    ซึ่งช่วยให้การพัฒนาเว็บมีประสิทธิภาพและความเร็วสูงขึ้น และยังมีฟีเจอร์ที่ช่วยในการทำ SEO ได้ดีขึ้น
    
    \item YOLOv8 เป็นระบบที่ใช้ในการพัฒนาโนโมเดลตรวจจับวัตถุความเร็วสูงแบบเวลาจริง ด้วยการเรียนรู้เชิงลึกและการมองเห็นคอมพิวเตอร์ 
    
    \item Figma เครื่องมือออกแบบเว็บไซต์ แอปพลิเคชัน โลโก้ และอื่น ๆ ทําให้นักออกแบบ UX/UI สะดวก มากขึ้น ผ่านการใช้ฟีเจอร์ต่าง ๆ 
    ซึ่งมีจุดเด่นอยู่ที่การใช้งานบนได้ทุกระบบปฏิบัติการ และยังมี Community ที่ผู้ใช้สามารถแชร์ไฟล์งาน Prototype หรือ Plug-in ต่าง ๆ 
    แล้วนําไปปรับใช้กับงานของตัว เองได้ 

    \item Linux เป็นระบบปฏิบัติการ (Operating System) ที่เป็น Open Source และเป็นพื้นฐานบนหลักการของ Unix ซึ่งถูกพัฒนาขึ้นโดย 
    Linus Torvalds ในปี ค.ศ. 1991 ซึ่งเป็นระบบปฎิบัตการที่เราจะนำมาใช้งาน 

    \item Kong เป็น API Gateway ที่ช่วยในการจัดการ API และการเชื่อมต่อระหว่างบริการต่าง ๆ ในระบบ ซึ่งช่วยเพิ่มความปลอดภัยและประสิทธิภาพในการทำงานของ API

    \item Docker เป็นแพลตฟอร์มที่ใช้ในการสร้าง จัดส่ง และรันแอปพลิเคชันในคอนเทนเนอร์ ซึ่งช่วยให้การพัฒนาและการนำแอปพลิเคชันไปใช้งานมีความยืดหยุ่นและรวดเร็ว

    \item Github Action เป็นเครื่องมือที่ใช้ในการทำ CI/CD (Continuous Integration/Continuous Deployment) บนแพลตฟอร์ม GitHub ซึ่งช่วยให้การทดสอบและการนำโค้ดไปใช้งานเป็นไปอย่างอัตโนมัติและมีประสิทธิภาพ

    \item Draw.io เป็นเครื่องมือออนไลน์ที่ใช้ในการสร้างไดอะแกรมและแผนภาพต่าง ๆ เช่น แผนภาพการไหล (flowchart) และแผนภาพสถาปัตยกรรมระบบ ซึ่งช่วยให้การออกแบบและสื่อสารข้อมูลเป็นไปอย่างมีประสิทธิภาพ

    \item Postman เป็นเครื่องมือที่ใช้ในการทดสอบ API ซึ่งช่วยให้นักพัฒนาสามารถส่งคำขอ (request) และดูผลลัพธ์ (response) ของ API ได้อย่างง่ายดาย

    \item PostgreSQL เป็นระบบจัดการฐานข้อมูลเชิงสัมพันธ์ (RDBMS) ที่มีความเสถียรและมีประสิทธิภาพสูง ซึ่งใช้ในการจัดการและเก็บข้อมูลในโครงการ

    \item MongoDB เป็นระบบจัดการฐานข้อมูลแบบ NoSQL ที่มีความยืดหยุ่นสูงและสามารถจัดการข้อมูลที่ไม่มีโครงสร้าง (unstructured data) ได้อย่างมีประสิทธิภาพ

    \item Redis เป็นฐานข้อมูลแบบ key-value ที่ทำงานในหน่วยความจำ (in-memory) ซึ่งมีความเร็วสูงและเหมาะสำหรับการจัดเก็บข้อมูลที่ต้องการการเข้าถึงอย่างรวดเร็ว

    \item Roboflow เป็นเครื่องมือที่สามารถใช้ทำการ Labeling ข้อมูล และสร้าง Dataset สำหรับการเทรนโมเดล Computer Vision ได้อย่างง่ายดาย
\end{enumerate}

\section{\ifenglish Project plan\else แผนการดำเนินงาน\fi}

\begin{plan}{10}{2024}{3}{2025}
    \planitem{10}{2024}{11}{2024}{ทดลองพัฒนา backend core logic ด้วย python}
    \planitem{11}{2024}{2}{2025}{เปลี่ยนมาพัฒนา backend core logic ด้วย typescript โดยใช้งาน elysia framework}
    \planitem{10}{2024}{3}{2025}{พัฒนา frontend ด้วย typescript}
    \planitem{10}{2024}{3}{2025}{พัฒนาแอปพลิเคชันด้วย flutter}
    \planitem{10}{2024}{12}{2024}{พัฒนาระบบ cloud service ด้วย azure ที่ใช้สำหรับประมวลผล}
    \planitem{10}{2024}{2}{2025}{พัฒนาโมเดลการเรียนรู้เชิงลึกด้วย YOLOv8}
    \planitem{10}{2024}{3}{2025}{พัฒนาระบบตรวจจับป้ายด้วยโมเดลที่พัฒนามาข้างต้น}
    \planitem{1}{2024}{2}{2025}{ทดสอบระบบและปรับปรุงแก้ไข}
    \planitem{2}{2025}{3}{2025}{ทำรายงาน คลิปวิดีโอ และโปสเตอร์}
    \planitem{3}{2025}{3}{2025}{นำเสนอผลงาน}
\end{plan}

% \section{\ifenglish Roles and responsibilities\else บทบาทและความรับผิดชอบ\fi}
% \begin{itemize}
%     \item นาย ชาญชล ภานุศุภนิรันดร์: ทำหน้าที่ศึกษาค้นคว้าข้อมูลที่จะนํามาใช้ในโครงงาน และจัดการเก็บข้อมูลและเชื่อม ส่วนต่อประสานเชิงประยุกต์ 
%     (API: Application Programming Interface) 
%     \item นาย ณัฐพงษ์ เทพพิทักษ์: ทําหน้าที่ศึกษาค้นคว้าข้อมูลที่จะนํามาใช้ในโครงงาน และพัฒนาโมไบล์แอปพลิเคชันถ่ายวิดีโอสำหรับการตรวจจับวัตถุ 
%     \item นาย ธนภัทร สมสิทธิ์: ทําหน้าที่ศึกษาค้นคว้าข้อมูลที่จะนํามาใช้ในโครงงาน และพัฒนาเว็บแอปพลิเคชันรายงานผลข้อมูลหลังจากการประมวลผลตรวจจับข้อมูล 
% \end{itemize}

\section{\ifenglish%
Impacts of this project on society, health, safety, legal, and cultural issues
\else%
ผลกระทบด้านสังคม สุขภาพ ความปลอดภัย กฎหมาย และวัฒนธรรม
\fi}

การพัฒนาระบบในการตรวจจับป้ายที่สามารถนำไปเก็บภาษีได้นั้น จะช่วยอำนวยความสะดวกให้สามารถจัดการได้ง่ายและสะดวกยิ่งขึ้น 
ซึ่งมีผลกระทบในด้านกฏหมายเพราะภาษีป้ายเป็นภาษีที่จัดเก็บจากป้ายที่ แสดงชื่อ ยี่ห้อ หรือเครื่องหมายที่ใช้ในการประกอบ การค้า หรือประกอบกิจการอื่นเพื่อหารายได้ หรือ 
โฆษณาการค้า ซึ่งในส่วนของการเสียนั้นก็ขึ้นอยู่กับประเภทของป้ายตามที่กฏหมายกำหนด และรายได้ที่ได้จากการจัดเก็บภาษีก็จะถูกนำไปพัฒนาบ้านเมืองต่อไป

\chapter{\ifenglish Background Knowledge and Theory\else ทฤษฎีที่เกี่ยวข้อง\fi}

การทําโครงงานเริ่มต้นด้วยการศึกษาค้นคว้าทฤษฎีที่เกี่ยวข้อง หรืองานวิจัย/โครงงานที่เคยมีผู้พัฒนาและนําเสนอไว้แล้ว ซึ่งเนื้อหาในบทนี้ก็จะเกี่ยวกับ
การอธิบายถึงทฤษฎีที่นำไปประยุกต์ใช้กับโครงงานนี้ เพื่ออำนวยให้ผู้อ่านทำความเข้าใจกับตัวระบบของโครงงานได้ง่ายขึ้น

\section{You Only Look Once Object Detection Algorithm (YOLO)}
YOLO \cite{yolo} เป็นอัลกอริทึมสำหรับการระบุบริเวณที่สนใจภายในภาพ และจำแนกประเภทของวัตถุบนแต่ละบริเวณแบบเวลาจริงเหมือนกับตัวจำแนกภาพปกติ 
โดยที่ภาพหนึ่งสามารถประกอบด้วยบริเวณที่สนใจหลายบริเวณ แล้วแต่ละบริเวณจะนำไปจำแนกวัตถุที่แตกต่างกันได้ ซึ่งทำให้เกิดความซับซ้อนสูงในการ
จำแนกภาพระหว่างการตรวจจับวัตถุ ต่างจากอัลกอริทึมตรวจจับวัตถุทั่วไปที่จะใช้อัลกอริทึมแบบ Two-stage Object Detection YOLO 
นั้นจะใช้แบบ Single-shot Object Detection แทน ซึ่งใช้การสแกนภาพแต่ละภาพเพียงครั้งเดียวสำหรับการพยากรตำแหน่งของวัตถุที่ต้อง
การจะตรวจจับ และเนื่องจากการประมวลผลภาพเพียงครั้งเดียวนั้น ส่งผลให้อัลกอริทึมดังกล่าวใช้ระยะเวลาในการประมวลผลต่ำ 
เหมาะกับการนำไปใช้แบบเวลาจริง แต่ก็แลกมากับข้อเสียที่ความแม่นยำในการตรวจจับภาพนั้นอาจไม่มากเท่าอัลกอริทึมแบบ Two-stage Object Detection 
โดยใช้เทคนิคการเรียนรู้แบบ Convolutional Neural Network ดังรูปที่ 2.1
\begin{figure}[ht]
  \begin{center}
  \includegraphics[scale=0.2]{resources/YOLO.png}
  \end{center}
  \caption[YOLO Architecture]{You Only Look Once Architecture}
  \label{fig:yolo architecture}
\end{figure}


\section{Object Relational Mapping (ORM)}
Object-Relational Mapping \cite{orm} เป็นการสร้างการสัมพันธ์ระหว่างฐานข้อมูลแบบ Relational กับโครงสร้างข้อมูลแบบ Object-Oriented 
ตามรูปที่ 2.2 ในการพัฒนาซอฟต์แวร์ เช่น เว็บแอปพลิเคชัน โดยที่ไม่ต้องเขียน SQL โดยตรงแต่สามารถใช้ภาษาโปรแกรมเพื่อจัดการกับข้อมูลแทน 
ซึ่งสามารถป้องกันการโจมตีแบบ SQL Injection ได้ ในกรณีที่กำหนดให้มีการเปลี่ยนแปลงในโครงสร้างข้อมูล 
คุณสมบัติหรือโครงสร้างข้อมูลในฐานข้อมูลจะถูกปรับเปลี่ยนตามในโครงสร้างของ Object ในโปรแกรม เป็นฐานข้อมูลแบบเสมือนในโปรแกรม 
โดยที่การจัดเก็บข้อมูลยังคงเป็นแบบ Relational เหมือนเดิม โดยไม่ต้องใช้ SQL Statements โดยตรง
\begin{figure}[ht]
  \begin{center}
  \includegraphics[scale=0.3]{resources/ORM.png}
  \end{center}
  \caption[Object Relational Mapping]{Object Relational Mapping}
  \label{fig:orm}
\end{figure}

\section{Model–View–Controller design pattern (MVC)}
Model-View-Controller \cite{mvc} เป็นรูปแบบโครงสร้างที่แยกแอปพลิเคชันออกเป็น 3 ส่วนหลักคือ: โมเดล (model), มุมมอง (view), 
และคอนโทรลเลอร์ (controller) แต่ละส่วนมีการสร้างขึ้นเพื่อจัดการด้านพัฒนาส่วนแอปพลิเคชันที่เฉพาะเจาะจง ตามรูปที่ 2.3 MVC 
เป็นหนึ่งในรูปแบบการพัฒนาเว็บตามมาตรฐานอุตสาหกรรมที่ถูกใช้บ่อยที่สุดเพื่อสร้างโครงงานที่สามารถเพิ่มและขยายขนาดในอนาคตได้ 
โดยที่ว่าเพื่อให้โปรแกรมนั้นดูเรียบง่ายต่อการแก้ไขจัดการ ซึ่งความหมายในแต่ละส่วนของ MVC นั้นได้แก่ 
\begin{enumerate}
  \item Model คือส่วนที่รับผิดชอบเกี่ยวกับข้อมูลและการประมวลผลทางด้านข้อมูลในแอปพลิเคชัน เช่น การเชื่อมต่อกับฐานข้อมูล การจัดการข้อมูล 
  และการประมวลผลทางข้อมูล เป็นต้น โดยที่ model มักจะเป็นตัวแทนของข้อมูลและสถานะของแอปพลิเคชัน
  \item View คือส่วนที่จะเป็นหน้าตาของโปรแกรมที่ผู้ใช้จะใช้งานจากส่วนนี้ ไม่ว่าจะเป็นการกรอกข้อมูล, ดูผลลัพธ์ หรือการมีปฏิสัมพันธ์กับผู้ใช้ 
  (User Interface) view จริง ๆ แล้วก็คือส่วนที่เรียกว่า GUI (Graphic User Interface) 
  \item Controller เป็นส่วนที่รับผิดชอบในการควบคุมและจัดการกับการกระทำที่เกิดขึ้นจากผู้ใช้งาน เช่น การรับข้อมูลจากผู้ใช้งาน, 
  การส่งข้อมูลไปยังโมเดลเพื่อประมวลผล, และการอัพเดตสถานะของ view ซึ่งโดยทั่วไปแล้ว controller จะเป็นตัวกลางที่เชื่อมต่อระหว่าง model 
  และ view โดยการควบคุมการทำงานของทั้งสอง 
\end{enumerate}
\begin{figure}[ht]
  \begin{center}
  \includegraphics[scale=0.3]{resources/MVC.png}
  \end{center}
  \caption[Model-View-Controller]{Model-View-Controller Design Pattern}
  \label{fig:mvc}
\end{figure}

\section{Hypertext Transfer Protocol (HTTP)}
HTTP (Hypertext Transfer Protocol) เป็นโปรโตคอลสื่อสารที่ใช้ในการส่งข้อมูลระหว่างเครื่องคอมพิวเตอร์บนเครือข่ายอินเทอร์เน็ต โดย HTTP 
มีหน้าที่เป็นตัวกลางในการร้องขอและส่งข้อมูลระหว่างเว็บไซต์ (web servers) และเบราว์เซอร์ (web browsers) หรือแอปพลิเคชันอื่น ๆ 
ที่ใช้เครือข่ายอินเทอร์เน็ต 
\begin{itemize}
  \item API (Application Programming Interface) เป็นชุดของกฎและโครงสร้างข้อมูลที่กำหนดโดยโปรแกรมคอมพิวเตอร์เพื่อให้แอปพลิเคชันอื่น ๆ 
  สามารถสื่อสารและทำงานร่วมกันได้ ในเชิงพื้นฐาน API เป็นวิธีที่แอปพลิเคชันใช้เรียกใช้ฟังก์ชันหรือการบริการที่ให้มาจากแหล่งข้อมูลหรือบริการ
  ซึ่งอาจเป็นเซิร์ฟเวอร์เว็บ ฐานข้อมูล หรือแหล่งข้อมูลอื่น ๆ โดยทั่วไป API จะรองรับการร้องขอและการตอบกลับโดยใช้ฟอแมตที่เป็นรูปแบบมาตรฐาน เช่น 
  JSON (JavaScript Object Notation) หรือ XML (Extensible Markup Language) 
\end{itemize}

\section{Docker}
Docker \cite{docker} เป็นเทคโนโลยีคอนเทนเนอร์แพลตฟอร์มที่ช่วยในการสร้างและทำการงานร่วมกับคอนเทนเนอร์อย่างมีประสิทธิภาพ ด้วย Docker 
ผู้ใช้สามารถแยกแยะและแพคเกจแอปพลิเคชันพร้อมกับสิ่งที่เกี่ยวข้องทั้งหมด เช่น ไฟล์ ระบบปฏิบัติการ ไลบรารี และสิ่งอื่น ๆ 
ลงในคอนเทนเนอร์ได้อย่างเรียบง่าย โดยมีโครงสร้างการทำงานตามรูปที่ 2.4 ผู้ใช้สามารถสร้าง และรันคอนเทนเนอร์ได้โดยง่าย นอกจากนี้ Docker 
ยังช่วยลดปัญหาเกี่ยวกับสภาพแวดล้อมและการติดตั้งโปรแกรมที่ซับซ้อน ทำให้การพัฒนาและการทำงานของโปรแกรมมีประสิทธิภาพมากขึ้น 
\begin{figure}[ht]
  \begin{center}
  \includegraphics[scale=0.3]{resources/Docker.png}
  \end{center}
  \caption[Docker Architecture]{Docker Architecture}
  \label{fig:docker}
\end{figure}

\section{Interactive Website}
Interactive website \cite{interactive-web} คือ เว็บไซต์ที่สามารถให้ผู้ใช้งาน communicate หรือ interact เช่น การแสดงความคิดเห็น การตอบโต้กับตัวเว็บ 
การได้รับผลจากการกระทําในเว็บ ในลักษระที่เป็นมิตรต่อผู้ใช้ โดยปัจจุบัน มักใช้ animation sound picture audio etc. ประกอบ 
เพื่อให้มีความสนุกสนานและเพิ่มการเข้าถึงได้ง่ายของผู้ใช้ ทั้งนี้อาจทําเพื่อเก็บข้อมูลหลังจากการใช้งานเว็บไซต์ได้อีกด้วย ซึ่งดีกว่าเว็บที่มีแต่ตัวอักษร หรือ 
การแสดงผลเฉย ๆ ที่ได้รับข้อมูลทางฝ่ายเดียวอย่างแน่นอน

% \subsubsection{Subsubsection 1 heading goes here}
% Subsubsection 1 text

% \subsubsection{Subsubsection 2 heading goes here}
% Subsubsection 2 text

% \section{Third section}
% Section 3 text. The dielectric constant\index{dielectric constant}
% at the air-metal interface determines
% the resonance shift\index{resonance shift} as absorption or capture occurs
% is shown in Equation~\eqref{eq:dielectric}:

% \begin{equation}\label{eq:dielectric}
% k_1=\frac{\omega}{c({1/\varepsilon_m + 1/\varepsilon_i})^{1/2}}=k_2=\frac{\omega
% \sin(\theta)\varepsilon_\mathit{air}^{1/2}}{c}
% \end{equation}

% \noindent
% where $\omega$ is the frequency of the plasmon, $c$ is the speed of
% light, $\varepsilon_m$ is the dielectric constant of the metal,
% $\varepsilon_i$ is the dielectric constant of neighboring insulator,
% and $\varepsilon_\mathit{air}$ is the dielectric constant of air.

% \section{About using figures in your report}

% % define a command that produces some filler text, the lorem ipsum.
% \newcommand{\loremipsum}{
%   \textit{Lorem ipsum dolor sit amet, consectetur adipisicing elit, sed do
%   eiusmod tempor incididunt ut labore et dolore magna aliqua. Ut enim ad
%   minim veniam, quis nostrud exercitation ullamco laboris nisi ut
%   aliquip ex ea commodo consequat. Duis aute irure dolor in
%   reprehenderit in voluptate velit esse cillum dolore eu fugiat nulla
%   pariatur. Excepteur sint occaecat cupidatat non proident, sunt in
%   culpa qui officia deserunt mollit anim id est laborum.}\par}

% \begin{figure}[h]
%   \centering

%   \fbox{
%      \parbox{.6\textwidth}{\loremipsum}
%   }

%   % To include an image in the figure, say myimage.pdf, you could use
%   % the following code. Look up the documentation for the package
%   % graphicx for more information.
%   % \includegraphics[width=\textwidth]{myimage}

%   \caption[Sample figure]{This figure is a sample containing \gls{lorem ipsum},
%   showing you how you can include figures and glossary in your report.
%   You can specify a shorter caption that will appear in the List of Figures.}
%   \label{fig:sample-figure}
% \end{figure}

% Using \verb.\label. and \verb.\ref. commands allows us to refer to
% figures easily. If we can refer to Figures
% \ref{fig:walrus} and \ref{fig:sample-figure} by name in the {\LaTeX}
% source code, then we will not need to update the code that refers to it
% even if the placement or ordering of the figures changes.

% \loremipsum\loremipsum

% % This code demonstrates how to get a landscape table or figure. It
% % uses the package lscape to turn everything but the page number into
% % landscape orientation. Everything should be included within an
% % \afterpage{ .... } to avoid causing a page break too early.
% \afterpage{
%   \begin{landscape}
%   \begin{table}
%     \caption{Sample landscape table}
%     \label{tab:sample-table}

%     \centering

%     \begin{tabular}{c||c|c}
%         Year & A & B \\
%         \hline\hline
%         1989 & 12 & 23 \\
%         1990 & 4 & 9 \\
%         1991 & 3 & 6 \\
%     \end{tabular}
%   \end{table}
%   \end{landscape}
% }

% \loremipsum\loremipsum\loremipsum

% \section{Overfull hbox}

% When the \verb.semifinal. option is passed to the \verb.cpecmu. document class,
% any line that is longer than the line width, i.e., an overfull hbox, will be
% highlighted with a black solid rule:
% \begin{center}
% \begin{minipage}{2em}
% juxtaposition
% \end{minipage}
% \end{center}

% \section{\ifenglish%
% \ifcpe CPE \else ISNE \fi knowledge used, applied, or integrated in this project
% \else%
% ความรู้ตามหลักสูตรซึ่งถูกนำมาใช้หรือบูรณาการในโครงงาน
% \fi
% }

% อธิบายถึงความรู้ และแนวทางการนำความรู้ต่างๆ ที่ได้เรียนตามหลักสูตร ซึ่งถูกนำมาใช้ในโครงงาน

% \section{\ifenglish%
% Extracurricular knowledge used, applied, or integrated in this project
% \else%
% ความรู้นอกหลักสูตรซึ่งถูกนำมาใช้หรือบูรณาการในโครงงาน
% \fi
% }

% อธิบายถึงความรู้ต่างๆ ที่เรียนรู้ด้วยตนเอง และแนวทางการนำความรู้เหล่านั้นมาใช้ในโครงงาน

\chapter{\ifproject%
\ifenglish Project Structure and Methodology\else โครงสร้างและขั้นตอนการทำงาน\fi
\else%
\ifenglish Project Structure\else โครงสร้างของโครงงาน\fi
\fi
}

\makeatletter

% \renewcommand\section{\@startsection {section}{1}{\z@}%
%                                    {13.5ex \@plus -1ex \@minus -.2ex}%
%                                    {2.3ex \@plus.2ex}%
%                                    {\normalfont\large\bfseries}}

\makeatother
%\vspace{2ex}
% \titleformat{\section}{\normalfont\bfseries}{\thesection}{1em}{}
% \titlespacing*{\section}{0pt}{10ex}{0pt}

\section{การใช้งานพื้นฐาน}
ในส่วนของโมไบล์แอปพลิเคชัน เป็นเครื่องมือที่จะจำเป็นต้องใช้งานกล้องและบันทึกพิกัดตำแหน่งทาง GPS อยู่ตลอดเวลาเพื่อทำการส่งรูปภาพ 
พร้อมกับพิกัดตำแหน่ง แล้วนำไประมวลผลในเซอร์วิสที่ได้ออกแบบเอาไว้ โดยที่เซอร์วิสดังกล่าวจะทำการประมวลผลรูปภาพเพื่อหาป้ายโฆษณาที่สามารถจัดเก็บภาษีได้ 
และหลังจากนั้นก็จะจัดเก็บลงฐานข้อมูลต่อไป 

ในส่วนของเว็บแอปพลิเคชัน จะเป็นส่วนของการแสดงผลข้อมูลที่ได้บันทึกมาได้ส่วนของโมไบล์แอปพลิเคชัน โดยจะแสดงในรูปแบบของหมุดในแผนที่ 
คล้าย ๆ กับการปักหมุดของ Google map โดยที่ในแต่ละหมุดสามารถกดเพื่อดูรายละเอียดต่าง ๆ ได้ เช่น พิกัดของหมุดนั้น 
และลักษณะรูปป้ายในตำแหน่งนั้นๆที่ได้บันทีกมาจากโมไบล์แอปฯ 

\section{การออกแบบระบบพื้นฐานของโครงงาน}
\subsection{Database Design}
ประกอบด้วย 4 ตารางดังรูปที่ 3.1 ได้แก่
\begin{enumerate}
  \item User table: เนื่องจากระบบต้องมีการ Authentication เพื่อเข้าใช้งานไม่ว่าจะเป็นทั้งส่วนของ โมไบล์แอปฯ หรือเว็บแอปฯ 
  ดังนั้นตารางนี้จึงจะใช้เก็บข้อมูลพืื้นฐานต่าง ๆ ที่จำเป็นต่อการยืนยันตัวตนทั้งหมด 
  \item Role table: ใช้ในการเก็บบทบาททั้งหมดที่มีของระบบ เช่น ผู้ดูแลระบบ ผู้สำรวจ และอื่น ๆ 
  \item Asset table: ใช้ในการเก็บข้อมูลที่ได้รับมาจาก โมไบล์แอปฯ ไม่ว่าจะเป็นตำแหน่งของรูป ชื่อของรูป และประเภทของ asset ที่ตรวจจับได้ 
  \item Asset type table: ใช้ในการเก็บประเภทของ asset ต่างๆที่ระบบสามารถตรวจจับได้ 
  \item Config table: ใช้เก็บการตั้งค่าพื้นฐานต่างๆเช่น ขอบเขตของแผนที่ 
\end{enumerate}

\begin{figure}[ht]
  \begin{center}
  \includegraphics[scale=0.8]{resources/ScreetnerDB.png}
  \end{center}
  \caption[Database Design]{Overall Database Design}
  \label{fig:database}
\end{figure}

% TODO: DELETE IF NECESSARY
\newpage
\subsection{System Design}
จากรูป จะอธิบายถึงโครงสร้างระบบของโครงงานงานนี้ในรูปแบบ Flow diagram เพื่อให้เข้าใจถึงโครงสร้างการทำงานพอสังเขป 
โดยที่ซอฟต์แวร์จะประกอบด้วย 3 ส่วนหลัก ๆ ได้ Mobile application ซึ่งจะทำงานตามรูปที่ 3.2 Web application ซึ่งจะทำงานตามรูปที่ 3.3 
และ Processing server โดยที่ลักษณะการทำงานร่วมกันระหว่างทั้งสามส่วนประกอบ แสดงตามรูปที่ 3.1 

\begin{figure}[ht]
  \begin{center}
  \includegraphics[scale=0.45]{resources/SystemDesign.png}
  \end{center}
  \caption[System Design]{Overall System Design}
  \label{fig:system design}
\end{figure}

% TODO: DELETE IF NECESSARY
\newpage
\subsection{Web Application Flow Diagram}
จากรูปที่ 3.3 จะอธิบายถึงลำดับการทำงานของเว็บแอปพลิเคชันในรูปแบบของ flow diagram เพื่อให้เข้าใจในลำดับการทำงานอย่างพอสังเขป 
พอหลังจากที่ได้เข้าระบบสู้หน้า dashboard จะมีตัวเลือกที่สามารถทำทำได้อยู่ 3 อย่างคือ filter เป็นการคัดกรองข้อมูลให้เหลือเพียงข้อมูลในช่วงเวลาที่เราต้องการ 
asset เป็นการกดที่รูปภาพเพื่อที่จะดูข้อมูลที่เกี่ยวกับ asset ดังกล่าว และ icon เป็นส่วนที่จะแสดงตัวเลือกเพิ่มเติมอีก 4 ทางเพื่อให้เราสามารถเลือกเข้าไปยังหน้าอื่นต่อไปได้

\begin{figure}[ht]
  \begin{center}
  \includegraphics[scale=0.65]{resources/WebsiteFlow.png}
  \end{center}
  \caption[Web Application Flow Diagram]{Web Application Flow Diagram}
  \label{fig:web-app flow design}
\end{figure}

% TODO: DELETE IF NECESSARY
\newpage
\subsection{Mobile Application Flow Diagram}
รูปที่ 3.4 จะอธิบายถึงลำดับการทำงานขอโมไบล์แอปพลิเคชันในรูปแบบของ flow diagram เพื่อให้เข้าใจในลำดับการทำงานอย่างพอสังเขป โดยพอผู้ใช้จะเริ่มเข้าใช้งานแอปพลิเคชัน 
ผู้ใช้จะต้องผ่านการเข้าสู่ระบบ ซึ่งมีขั้นตอนการทำงานดังรูปที่ 3.5 เพื่อเป็นการยืนยันตัวตน หลังจากที่ได้เข้าสู่แอปพลิเคชันเรียบร้อยแล้ว 
ผู้ใช้งานจะสามารถเริ่มสตรีมวิดีโอเพื่อทำการส่งรูปภาพในช่วงเวลาหนึ่งพร้อมแนบตำแหน่งพิกัดในช่วงเวลาดังกล่าวไปยังเซอร์วิสที่ได้จัดเตรียมเอาไว้อยู่ตลอดเวลาที่ทำการสตรีม 
เพื่อให้ทางเซอร์วิสทำการคืนค่าออกมาว่าในตำแหน่งนี้จะมี asset อยู่เท่าไหร่ โดยจะมีขั้นตอนการทำงานดังรูปที่ 3.6 
ซึ่งสิ่งที่คืนค่ามาทุกครั้งนั้นจะเอามาจัดเก็บเอาไว้บนมือถือชั่วคราวและยังไม่ได้ทำการบันทึกข้อมูลลงไปในฐานข้อมูลเพื่อให้ผู้ใช้งานสามารถดูข้อมูลได้ตลอดเวลาว่าปัจจุบันมี 
asset อยู่เท่าไหร่จนจบการทำงาน และในตอนท้ายของการทำงานผู้ใช้สามารถที่จะเลือกได้ว่าจะทำการสตรีมต่ออีกครั้งหรือไม่ หากไม่ทำการสตรีมต่อ 
ผู้ใช้งานต้องเลือกว่าจะทำการส่งข้อมูลทั้งหมดที่ได้มานั้นไปยังฐานข้อมูลหรือไม่
\begin{figure}[ht]
  \begin{center}
  \includegraphics[scale=0.5]{resources/MobileAppFlow.png}
  \end{center}
  \caption[Mobile Application Flow Diagram]{Mobile Application Flow Diagram}
  \label{fig:mopile-app flow design}
\end{figure}

\begin{figure}[ht]
  \begin{center}
  \includegraphics[scale=0.6]{resources/LoginFlow.png}
  \end{center}
  \caption[Login Flow Diagram]{Login Flow Diagram}
  \label{fig:login flow design}
\end{figure}

\begin{figure}[ht]
  \begin{center}
  \includegraphics[scale=0.6]{resources/TransitionFlow.png}
  \end{center}
  \caption[Transition Flow Diagram]{Transition Flow Diagram}
  \label{fig:transition flow design}
\end{figure}

% TODO: UNCOMMENT THE FOLLOWING
% \chapter{\ifproject%
% \ifenglish Experimentation and Results\else การทดลองและผลลัพธ์\fi
% \else%
% \ifenglish System Evaluation\else การประเมินระบบ\fi
% \fi}

% TODO: DELETE THE FOLLOWING
\chapter{\ifenglish System Evaluation\else การประเมินระบบ\fi}

\section{การประเมินประสิทธิภาพซอฟต์แวร์}
ทดสอบประสิทธิภาพซอฟต์แวร์โดยจะมีการแบ่งส่วนในการทดสอบออกเป็นส่วน ๆ เพื่อให้รู้ว่าในแต่ละส่วนของซอฟต์แวร์ของเรานั้น 
ทำงานได้อย่างมีประสิทธิภาพหรือไม่ จึงสามารถแบ่งออกการประเมินได้เป็นดังนี้ 
\begin{enumerate}
    \item Classification model - เป็นการทดสอบเพื่อประเมินและตรวจสอบความเร็วในการประมวลผลเพื่อทำการ classify 
    ว่า object ใดเป็นป้ายที่สามารถจัดเก็บภาษีได้ รวมถึงในเรื่องของความแม่นยำในการ classify  
    \item Response time - เป็นการทดสอบเพื่อประเมินในเรื่องของความเร็วในการรับส่งข้อมูลระหว่าง client กับ application server  
\end{enumerate}

\section{การประเมินความพึงพอใจในการใช้งานระบบ}
ทดสอบความพึงพอใจในการใช้งานจะมีการแบ่งออกเป็นสองส่วน คือส่วนของแอปพลิเคชันในโทรศัพท์มือถือ กับส่วนของเว็บแอปพลิเคชัน 
โดยจะมีเกณฑ์การให้คะแนนอยู่ที่ 1 ถึง 5 โดยจะมีการให้คะแนนในเรื่องดังต่อไปนี้ 
\begin{enumerate}
    \item ความง่ายต่อการใช้งานของแอปพลิเคชัน 
    \item ความสะดวกในการใช้งานในตอนเริ่มต้นของแอปพลิเคชัน 
    \item ความดึงดูดในการใช้งานของแอปพลิเคชัน        
    \item ประโยชน์ที่มีของแอปพลิเคชัน 
\end{enumerate}
โดยที่ทั้ง 4 ข้อเป็นพิจราณาจากแนวคิดตาม The Four Elements of User Experience \cite{uxquantification} ที่ประกอบไป ด้วย 
\begin{enumerate}
    \item Usability ความใช้ง่ายในการใช้งาน เกี่ยวข้องกับสามารถในการใช้งาน รวมไปถึงความเหมาะสมการใช้งานกับผู้งานใช้ 
    \item Adaptability ความสามารถในงานปรับตัว กล่าวถึงระดับความยากง่ายของการใช้งานตั้งแต่จุดเริ่มต้น จนถึงจุดสิ้นสุดของระบบ โดยที่ผู้งานสามารถใช้งานได้อย่างคล่องแคล่ว 
    \item Desirability ความพึงพอใจ คือเมื่อใช้งานแล้วผู้ได้รับประสบการณ์ที่ดีในจากใช้งานของระบบ 
    \item Value คุณค่าของระบบ คือระบบที่ผู้ใช้เข้ามาใช้งานมีความสอดคล้องกับความต้องการของผู้ใช้ 
\end{enumerate}
% \ifproject
% \chapter{\ifenglish Conclusions and Discussions\else บทสรุปและข้อเสนอแนะ\fi}

\section{\ifenglish Conclusions\else สรุปผล\fi}

โครงงานนี้ประสบความสำเร็จในการพัฒนาระบบครบทั้ง 3 ส่วนตามวัตถุประสงค์ที่กำหนดไว้ ประกอบด้วย (1) ส่วน web front-end สำหรับผู้ใช้งานทั่วไปและผู้ดูแลระบบ ที่ช่วยให้การเข้าถึงและจัดการข้อมูลเป็นไปอย่างมีประสิทธิภาพ (2) ส่วน mobile application สำหรับการตรวจจับและประมวลผลป้ายจราจรแบบเรียลไทม์ และ (3) ส่วน back-end ที่รองรับการประมวลผลข้อมูล การจัดเก็บ และการเชื่อมต่อระหว่างส่วนต่างๆ ของระบบ แม้ว่าระบบจะสามารถทำงานได้ตามเป้าหมายหลัก แต่ยังพบข้อจำกัดบางประการที่สำคัญ ดังนี้

\begin{itemize}
    \item ความยาววีดีโอต่อ 1 เซสชั่นไม่ควรเกิน 30 นาที
    \item ระบบตรวจจับป้ายอาจมีการทำงานผิดพลาด เนื่องจากความแม่นยำของตัวโมเดลไม่ได้สูงมาก
\end{itemize}

\section{\ifenglish Challenges\else ปัญหาที่พบและแนวทางการแก้ไข\fi}

\begin{itemize}
    \item การทำงานของระบบตรวจจับป้ายยังมีความแม่นยำที่ต่ำ เนื่องจากการเตรียมข้อมูลสำหรับการเรียนรู้ของโมเดลไม่เพียงพอ ดังนั้นวิธีการแก้ไขคือการเพิ่มข้อมูลสำหรับการเรียนรู้ของโมเดลและปรับปรุงโมเดลให้มีความแม่นยำมากขึ้น
    \item ความยาววิดีโอที่ไม่ควรเกิน 30 นาที นั้นสืบเนื่องมาจากระบบ auto scale ที่ได้ออกแบบมานั้นมีระบนยะเวลาการทำงานต่อ 1 เซสชั่นประมาณ 1 ชั่วโมง ซึ่งควารมยาวของการประมวลผลจะใช้เวลาเป็น 2 เท่าของความยาววิดีโอ ดังนั้นความยาววิดีโอจึงไม่ควรที่จะเกิน 30 นาที และวิธีการแก้ไขคือการปรับปรุงระบบ auto scale ให้สามารถทำงานได้เร็วขึ้น
\end{itemize}

\section{\ifenglish%
Suggestions and further improvements
\else%
ข้อเสนอแนะและแนวทางการพัฒนาต่อ
\fi
}

ข้อเสนอแนะเพื่อพัฒนาโครงงานนี้ต่อไป มีดังนี้
% \fi

\bibliography{ScreetnerReport}

% \ifproject
% \normalspacing
% \appendix
% \include{chapters/appendix}

%% Display glossary (optional) -- need glossary option.
% \ifglossary\glossarypage\fi

%% Display index (optional) -- need idx option.
% \ifindex\indexpage\fi

% \begin{biosketch}
% \begin{center}
%   \includegraphics[width=1.5in]{mugshot.jpg}
% \end{center}
% Your biosketch goes here. Make sure it sits inside
% the \texttt{biosketch} environment.
% \end{biosketch}
% \fi % \ifproject
\end{document}
